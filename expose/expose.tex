\documentclass[a4paper, 12pt]{scrartcl}

\usepackage{setspace}
% \onehalfspacing

\usepackage{acronym}
\newacro{SIL}{Safety Integrity Level}
\acrodefplural{SIL}{Safety Integrity Levels}
\newacro{ETCS}{European Train Control System}
\newacro{SPS}[PLC]{Programmable Logic Controller}
\acrodefplural{SPS}[PLCs]{Programmable Logic Controllers}
\newacro{DDS}{Data Distribution Service}
\newacro{OS}{Operating System}

\begin{document}


% Titel
\begin{center}
  \Huge{Expos\'{e} Master's Thesis}\\
\Large{Hendrik Tjabben}
\end{center}

% Einleitung
% \section*{Abstract}

\section*{Introduction}
% Context CHECK
% Status Quo CHECK
% Problem and Challenges CHECK

% HALF PAGE
Embedded systems, which are deployed in safety-critical environments, must comply with suitable \acp{SIL} to ensure the system's functionality.
Nowadays, such safety-critical systems are demanded within various industries, including aero, medical and automotive applications.
In order to verify the system's safety, its correspondance with the specified \ac{SIL} is ensured during an approval process.
This approval process requires a lot of time and money for large systems with a high \ac{SIL}.
In addition, the approval process needs to be repeated every time the system is updated.  One way of reducing costs is to minimize the system's parts with high \acp{SIL}.
Other approaches to build safety-critical systems consider distributed designs, consisting of modular parts with different safety-demands.
The individual parts of this system are interconnected through a protocol that must not violate the system's safety requirements.
Such distributed systems reduce costs for approval by reducing the size of components to check and allow individual modules to be replaced with other approved modules.

\section*{Aim of the project}
% Proposed Solution CHECK
% Academic view

% HALF A PAGE
In the course of this master thesis, such a distributed system is build and validated against a subset of \ac{ETCS}, namely balise-group-to-train communication.
Therefore, the system's modules exchange data following the \ac{DDS} standard's publish-subscribe pattern.
Upon the test system, characteristics such as licenseability, reliability, maintainablity and performance are examined.
Further, it is evaluated which parts of the system can run on a low safety-level of \ac{SIL}2 and what subset of \ac{DDS} is required to still guarantee an overall system safety-level of \ac{SIL}4.

\subsection*{Experimental Setup}
The experimental setup will comprise of a PC, four \acp{SPS}, an interconnect, a middleware layer running the \ac{DDS} standard and a simulation software.
The PC functions as the main computation unit but has a low \ac{SIL} of \ac{SIL}2.
However, the \acp{SPS} feature a secure \ac{SIL}4 \ac{OS} and are used for redundancy computing to safeguard the main PC's computations.
All \acp{SPS} are connected via an interconnect to the main PC.
By swapping out safety critical computations to the redundant working \acp{SPS} and sending data between individual components using the dependable real-time data exchange standard \ac{DDS}, the entire system can be raised to a \ac{SIL}4.

The simulation software imitates trackside balise components and their corresponding data telegrams, which are received and evaluated by the PC and the four \acp{SPS}.
In order to facilitate a realistic test environment, the balise telegrams as well as the telegram evaluations are based upon \ac{ETCS}.
Different balise telegrams are triggered on predefined positions along the track.
Therefore, the train's position is monitored using a digital rail network plan, as well as position sensoring such as GPS, traveled distance and further technology.

\section*{Related Work}
%HALF A PAGE
TODO

% DDS

% ETCS Balisen

\section*{Work and Risk Analysis}
% HALF A PAGE

The project can be subdevided into four parts, namely a train and track simulator, a hardware setup, an on-board processing system as well as a testing and evaluation phase.
This work's main part is made up by the testing and evaluation phase, since this thesis' aim it to experiment and evaluate the system's configuration and characteristics.
The on-board processing system comprises of data exchange with \ac{DDS}, decision making and redundancy computation.
It lays the foundation for the evaluation phase and is expected to take up the most time in this project.
Connecting the hardware is an essential part in this project but is unlikely to consume much time.
Simulating a train and track-side balises is choosen to be the show-case of this work.
Because of \ac{ETCS}'s complexity, building such a simulator from scratch is likely to take much resources.
It's preferred to either use existing simulators that provide the required packages or limit to very basic and time-periodic messages at first and postpone the implementation of a train simulation to the end of this project.
For evaluating the system, it is not mandatory that the analyzed messages conform to the \ac{ETCS} specification, so that correspondig to \ac{ETCS} should not considered to be a main requirement of this project.

\end{document}
