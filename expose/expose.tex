\documentclass[a4paper, 12pt]{scrartcl}
\usepackage[utf8]{inputenc}
\usepackage[english]{babel}

\usepackage{hyperref}

\usepackage[left=2.5cm, right=2.5cm, top=2.0cm]{geometry}

\usepackage{setspace}
% \onehalfspacing

\usepackage{acronym}
\newacro{DDS}{Data Distribution Service}
\newacro{ETCS}{European Train Control System}
\newacro{ERTMS}{European Rail Traffic Management System}
\newacro{ETCS OBU}{European Train Control System Onboard Unit}
\newacro{GNSS}{Global Navigation Satellite System}
\newacro{GNSSINS}[GNSS/INS]{Global Navigaion Satellite System/Inertial Navigation System}
\newacro{INS}{Inertial Navigation System}
\newacro{IOT}[IoT]{Internet of Things}
\acrodefplural{RBC}{Radio Block Centers}
\newacro{MA}{Movement Authority}
\acrodefplural{MA}{Movement Authorities}
\newacro{OMG}{Object Management Group}
\newacro{OS}{Operating System}
\newacro{RBC}{Radio Block Center}
\newacro{SIL}{Safety Integrity Level}
\acrodefplural{SIL}{Safety Integrity Levels}
\newacro{SPS}[PLC]{Programmable Logic Controller}
\acrodefplural{SPS}[PLCs]{Programmable Logic Controllers}
\newacro{SSP}{Static Speed Profile}
\acrodefplural{SSP}{Static Speed Profiles}

\begin{document}


% Titel
\begin{center}
  \Huge{Master's Thesis Expos\'{e}}\\
  \large{\textsc{Hendrik Tjabben}}
\end{center}

\section*{Introduction}
Embedded systems, which are deployed in safety-critical environments, must comply with certification requirements to demonstrate the system's functional safety and integrity.
Nowadays, such safety-critical systems are demanded within various industries, including aero, medical, transportation and automotive applications.
In order to verify the system's safety, its compliance with the required assurance level is ensured during an approval process.
The conducted verification, validation and certification steps are very time- and cost-intensive.
Moreover, the approval process needs to be repeated every time the system is updated.

\section*{Aim of the project}
In the course of this master thesis, a distributed system architecture that combines quality characteristics, such as simplicity, modularity, control\-ler-level partitioning, and distributed processing, with dynamic reconfiguration and data-driven designs, is assessed and demonstrated.
This architecture allows applications to be broken down in modules with distinct safety and integrity requirements.
In order to prove the approach, such a distributed system is build and validated against a core functionality of the \ac{ETCS OBU}.
Therefore, essential track-to-train messages from balises and \acp{RBC} are decoded, \acp{MA} and \acp{SSP} are calculated and the vehicle's speed is supervised.
The functional modules will be distributed across a modular system which is interconnected using the \ac{OMG} \ac{DDS} standard's publish-subscribe pattern.
Upon the test system, characteristics such as certifiability - with a focus modularity and the identification of a sufficient \ac{DDS} subset - maintainability and performance are examined.
Further, a high-level safety case analysis will be provided in order to identify functions that could be mapped to nodes with lower \acp{SIL}.

\section*{Experimental Setup}
The experimental setup will comprise of a PC, four \acp{SPS}, an interconnect, a middleware layer running the \ac{DDS} standard and a simulation software.
The entire computation is done by the PC, which functions as the main computation unit but has no safety guarantees at all.
However, the \acp{SPS} feature a high \ac{SIL} \ac{OS} and are used for redudancy computation by only reprocessing the workload's safety-critical parts.
Synchronizing the \ac{SPS}'s results with the main PC supervises its work and wraps it into a safe envelope.
All \acp{SPS} are connected via an interconnect to the main PC.
By swapping out safety critical computations to the redundant working \acp{SPS} and sending data between individual components using the dependable real-time data exchange standard \ac{DDS}, the entire system can be raised to a high \ac{SIL}.

The simulation software imitates trackside balise as well as \ac{RBC} components and their corresponding data telegrams and messages, which are received and evaluated by the PC and the four \acp{SPS}.
In order to facilitate a realistic test environment, the balise and \ac{RBC} telegrams as well as the functional and operational principles are based on \ac{ETCS} Subset-026.

The simulation software should also be deployable to a lab vehicle's on-board system.
In that case, it will use positioning information based on \ac{GNSSINS} and odometry data of the lab vehicle in order to provide the system with virtual balise information.

\section*{Related Work}
\paragraph{\ac{ETCS}} is a part of \ac{ERTMS} and comprises of the signalling and control components.
It lays the foundation for further work on automatic train operation by removing the need of manual linesight signal observation in its most advanced form.
In this work, the \ac{ETCS} Subset-026 is used.
\vspace{-0.4cm}
\paragraph{\ac{DDS}} is a data-centric connectivity standard and middleware protocol by the \ac{OMG} for machine-to-machine communication using a publish-subscribe pattern.
It allows reliable and scalable communication architectures for critical \ac{IOT} applications.
\vspace{-0.4cm}
\paragraph{Cyclone DDS} is an implementation of the \ac{DDS} standard developed by ADLINK.
The project is open source and licensed by the Eclipse Public License - v 2.0.
In this project, Cyclone DDS will be used for the middleware \ac{DDS} layer's implementation.
\vspace{-0.4cm}
\paragraph{Designing Data-Intensive Applications} by Martin Kleppmann can provide good insights into designing a distributed system for redundancy computing.
Problems that can occur withing a distributed systems for redundant compting include unreliable networks, partial failures and finding a system-wide consensus.

\section*{Work and Risk Analysis}
The project can be subdevided into four parts, namely a train and track simulator, a hardware setup, an on-board processing system as well as a testing and evaluation phase.
This work's main part is made up by the testing and evaluation phase, since this thesis' aim is to experiment and evaluate the system's configuration and characteristics.
The on-board processing system is comprised of data exchange with \ac{DDS}, decision making and redundancy computation.
It lays the foundation for the evaluation phase and is expected to take up the most time in this project.
Connecting the hardware is an essential part in this project but is unlikely to consume much time.
Simulating a train and track-side balises is choosen to be the show-case of this work.
Because of \ac{ETCS}'s complexity, building such a simulator from scratch is likely to take much resources.
It's preferred to either use existing simulators that provide the required packages or limit to very basic and time-periodic messages at first and postpone the implementation of a train simulation to the end of this project.
For evaluating the system, it is not mandatory that the analyzed messages conform to the \ac{ETCS} specification, so that adhering to \ac{ETCS} should not considered to be a main requirement of this project.

\end{document}
