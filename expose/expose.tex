\documentclass[a4paper, 12pt]{scrartcl}

\usepackage{setspace}
\onehalfspacing

\usepackage{acronym}
\newacro{SIL}{Safety Integrity Level}
\acrodefplural{SIL}{Safety Integrity Levels}
\newacro{ETCS}{European Train Control System}
\newacro{SPS}[PLC]{Programmable Logic Controller}
\acrodefplural{SPS}[PLCs]{Programmable Logic Controllers}
\newacro{DDS}{Data Distribution Service}
\newacro{OS}{Operating System}

\begin{document}


% Titel
\begin{center}
  \Huge{Expos\'{e} Master's Thesis}\\
\Large{Hendrik Tjabben}
\end{center}

% Einleitung
% \section*{Abstract}

\section*{Introduction and Aim of the project}
Embedded systems, which are deployed in safety-critical environments such as the railway, must comply with suitable \acp{SIL} to ensure the system's functionality.
In order to verify the system's safety, its correspondance with the specified \ac{SIL} is ensured during an approval process.
This approval process requires a lot of time and money for large systems with a high \ac{SIL}.
In addition, the approval process needs to be repeated every time the system is updated.
One way of reducing costs is to minimize the system's parts with high \acp{SIL}.
Other approaches consider distributed designs, consisting of modular parts with different safety-demands, to build safety-critical systems.
The individual parts of this system are interconnected through a protocol that must not violate the system's safety requirements.
In the course of this master thesis, such a distributed system is build and validated using an example of on-board balise protocol evaluation based on the \ac{ETCS}.
Therefore, the system's parts exchange data following the \ac{DDS} standard's publish-subscribe pattern.
Upon the test system, characteristics such as licenseability, reliability, maintainablity and performance are examined.
Further, it is evaluated which parts of the system can run on a low safety-level of \ac{SIL}2 and what subset of \ac{DDS} is required to still guarantee an overall system safety-level of \ac{SIL}4.

\subsection*{Experimental Setup}
The experimental setup will comprise of a PC, four \acp{SPS}, an interconnect, a middleware layer running the \ac{DDS} standard and a simulation software.
The PC functions as the main computation unit but has a low \ac{SIL} of \ac{SIL}2.
However, the \acp{SPS} feature a secure \ac{SIL}4 \ac{OS} and are used for redundancy computing to safeguard the main PC's computations.
All \acp{SPS} are connected via an interconnect to the main PC.
By swapping out safety critical computations to the redundant working \acp{SPS} and sending data between individual components using the dependable real-time data exchange standard \ac{DDS}, the entire system can be raised to a \ac{SIL}4.

The simulation software imitates trackside balise components and their corresponding data telegrams, which are received and evaluated by the PC and the four \acp{SPS}.
In order to facilitate a realistic test environment, the balise telegrams as well as the telegram evaluations are based upon \ac{ETCS}.
Different balise telegrams are triggered on predefined positions along the track.
Therefore, the train's position is monitored using a digital rail network plan, as well as position sensoring such as GPS, traveled distance and further technology.


\section{Roadmap}


\end{document}
