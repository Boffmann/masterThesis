\documentclass{scrreprt} % <= Druckversion: "scrbook" / Bildschirmversion: "scrreprt"
\usepackage[english,bibtex]{osm-thesis} % <= Sprache der Arbeit ("ngerman"/"english"), Biblatex-Backend ("bibtex"/"biber")

\usepackage{amssymb}
\usepackage{tabularx}
\usepackage{xspace}

% ABOUT
\newcommand{\hpitype}{Masterarbeit}
\newcommand{\hpiauthor}{Hendrik Tjabben}
\newcommand{\hpititle}{V 0.6}
\newcommand{\hpititleother}{TODO} % <= das Studienreferat verlangt einen deutschen UND englischen Titel
\newcommand{\hpisupervisor}{Prof.\,Dr.\,Andreas Polze, Robert Schmid, Lukas Pirl, Arne Boockmeyer}
\newcommand{\hpichair}{Professur für Betriebssysteme und Middleware}
\newcommand{\hpiexternalsupervisor}{Jakob Gärtner}
\newcommand{\hpiexternal}{Railergy}
\newcommand{\hpidate}{\today}


\newcommand{\etal}{\textit{et al.}\xspace}
\newcommand{\abr}{\gls*}
\newcommand{\abrpl}{\glspl*}

\usepackage[acronym]{glossaries}
\newacronym{API}{API}{Application Programming Interface}
\newacronym{ATO}{ATO}{Automatic Train Operation}
\newacronym{CPU}{CPU}{Central Processing Unit}
\newacronym{CT}{CT}{Continuous Time}
\newacronym{DCPS}{DCPS}{Data-Centric Publish-Subscribe}
\newacronym{DDS}{DDS}{Data Distribution Service}
\newacronym{DE}{DE}{Discrete Event}
\newacronym{ECC}{ECC}{Error Checking and Correction}
\newacronym{ERTMS}{ERTMS}{European Rail Traffic Management System}
\newacronym{ETCS}{ETCS}{European Train Control System}
\newacronym{EU}{EU}{Execution Unit}
\newacronym{FCR}{FCR}{Fault Containment Region}
\newacronym{FDI}{FDI}{failure detection and identification}
\newacronym{GSMR}{GSM-R}{Global System for Mobile Communication - Railway}
\newacronym{IDL}{IDL}{Interface Definition Language}
\newacronym{IOT}{IoT}{Internet of Things}
\newacronym{IP}{IP}{Internet Protocol}
\newacronym{MA}{MA}{Movement Authority}
\newacronym{MOON}{MooN}{M-out-of-N}
\newacronym{OMG}{OMG}{Object Management Group}
\newacronym{OS}{OS}{Operating System}
\newacronym{PFD}{PFD}{probability of failure on demand}
\newacronym{PLC}{PLC}{Programmable Logic Controller}
\newacronym[plural=QoS's,firstplural=Quality of Services (QoS)]{QOS}{QoS}{Quality of Service}
\newacronym{RAM}{RAM}{Random Access Memory}
\newacronym{RBC}{RBC}{Radio Block Center}
\newacronym{RPC}{RPC}{Remote Procedure Call}
\newacronym{SDN}{SDN}{Software Defined Networking}
\newacronym{SIL}{SIL}{Safety Integrity Level}
\newacronym{TMR}{TMR}{Triple Modular Redundancy}
\newacronym{TTL}{TTL}{Time to Live}
\newacronym{UAV}{UAV}{Unmanned Aerial Vehicle}


% DOCUMENT
\bibliography{bibliography}

\begin{document}

	% Einband
	\pagenumbering{alph}
	\ifisbook\include{content/coverpage}\fi
	\ifisbook\cleardoubleemptypage\fi

	% (Haupt-)Titelseite, Abstract, ggf. Danksagung & Inhaltsverzeichnis
	\pagenumbering{roman}
	\include{content/titlepage}
	% \ifisbook\cleardoubleemptypage\fi% => Wenn die Arbeit auf Deutsch verfasst wurde, verlangt das Studienreferat KEINEN englischen Abstract

% % englischer Abstract
\null\vfil
\begin{otherlanguage}{english}
\begin{center}\textsf{\textbf{\abstractname}}\end{center}
%
\noindent \glsentryfull{ETCS} on-board units are safety-critical systems whose reliability plays a vital role in the integrity of railway operation.
While redundancy is a typical method for increasing fault tolerance, reliability, and safety, it also adds a communication and computation overhead.
Further, the development and maintenance of highly safety-critical applications are resource-intensive.
This thesis provides background and implementation of a redundant and fault-tolerant ETCS on-board system using real-time \glsentryfull{DCPS} machine-to-machine communication and consensus-based voting.
Besides behavioral concepts, such as a leader election and decision-making algorithm, the work exposes global system state and system recovery mechanisms in distributed systems.
The implementation's functionality, safety, and reliability are evaluated based on a subset of ETCS in a simulated environment.
The results confirm DCPS concepts applicable for solving the communication and computation overhead of distributed and redundant computation in a real-time and safety-critical environment.
Furthermore, the findings also apply to an architecture of general-purpose Programmable Logic Controllers (PLCs).
Thereby, this work facilitates the development of safe and more cost-efficient on-board systems for future ETCS applications.

\end{otherlanguage}
\vfil\null

\iffalse

% => Wenn die Arbeit auf Englisch verfasst wurde, verlangt das Studienreferat einen englischen UND deutschen Abstract (der dt. Abstract kann dann ggf. auch ans Ende der Arbeit)

% deutsche Zusammenfassung
\null\vfil
\begin{otherlanguage}{ngerman}
\begin{center}\textsf{\textbf{\abstractname}}\end{center}

\noindent Lorem ipsum dolor sit amet, consetetur sadipscing elitr, sed diam nonumy eirmod tempor invidunt ut labore et dolore magna aliquyam erat, sed diam voluptua. At vero eos et accusam et justo duo dolores et ea rebum. Stet clita kasd gubergren, no sea takimata sanctus est Lorem ipsum dolor sit amet. Lorem ipsum dolor sit amet, consetetur sadipscing elitr, sed diam nonumy eirmod tempor invidunt ut labore et dolore magna aliquyam erat, sed diam voluptua. At vero eos et accusam et justo duo dolores et ea rebum. Stet clita kasd gubergren, no sea takimata sanctus est Lorem ipsum dolor sit amet.

\end{otherlanguage}
\vfil\null

\fi



	%\ifisbook\cleardoubleemptypage\fi\vspace*{\fill}
\begin{center}\textsf{\textbf{Acknowledgment}}\end{center}

\noindent Throughout the writing of my thesis, I have received great support and assistance.
\\

\noindent First of all, I would like to thank my supervisor, Prof. Andreas Polze, whose expertise and systematic methodology were inevitably for this work's structure and success.
\\

\noindent I would further like to acknowledge my tutors Robert Schmid, Lukas Pirl, and Arne Boockmeyer, who supported me with their valuable advice.
Besides all the tools I needed, you gave me indispensable and regular feedback to choose the right direction and successfully complete my thesis.
\\

\noindent I would also like to thank Jakob Gärtner for his dedication, support, and insights.
Jakob, you not only provided the hardware for the project but also were decisive in formulating the research question.
Without you, this work would not have been possible.
\\

\noindent In addition, I would like to thank my family, friends, and partner.
You were always there for me during a time, where social contacts must be limited to a minimum.
Thank you for every conversation and any welcome distraction. 


\vspace*{\fill}
	\tableofcontents
	\cleardoublepage

	% Textteil
	\pagenumbering{arabic}
    \chapter{Introduction}

%https://www.myperfectwords.com/blog/thesis-writing/writing-a-thesis-introduction

% Attention grabbing hook statement
Railway operation is undergoing rapid changes.
%
% General introduction to topic through general statements
As digitalization progresses, a trend towards the \abr{IOT} is developing and devices such as vehicles and machines are interconnected~\cite{RailwayDigitalization}.
With the increasing traffic and the importance of rail traffic in climate issues, digitization offers a way to deal with growing complexity.

On the one hand, studies have shown that 41\% of all railway accidents were caused by human failures~\cite{StudyRailwayAccidents}.
Improvements in computation, communication technologies, and railway applications laid the foundation for safe and reliable \abr{SIL} 4 certified \abr{ATO} systems with a small \abr{PFD}~\cite{SallekSIL}.
Further, the standardization of signaling and control components in the \abr{ETCS} facilitated train operation across borders and increases the significance of \abr{ATO} systems~\cite{YIN2017RNDofATO}.

On the other hand, the automation of railway operations comes with higher development and acquisition costs, as well as with ethical issues.
System designers are required to specify the system's reaction under specific conditions and balance the safety of humans, such as passengers and road users, and of resources, such as costs and time~\cite{EthicsInSafety}.

Hence, with the railway being a safety-critical system, any component that is applied in the \abr{ETCS} context must comply with strict certification requirements to ensure its safety and reliability.
The compliance of system requirements is ensured during an approval process.
The thereby applied verification, validation and certification steps are time- and cost-intensive, which adds to the already high development costs.
Moreover, the approval process needs to be repeated every time the system gets updated.
Increasing costs for \abr{ETCS} systems could lead to a situation where it is not profitable to equip certain regions, connections, or train models with \abr{ETCS}.
\\

Distributed computation is a way to reduce the costs of repeated approval processes and increase its scalability by arranging autonomous computing elements that perform a subset of the system's requirements, each~\cite{DistributedSafety2020}.
However, distributed systems also introduce new challenges such as coping with the communication overhead and dealing with node-, network-, and computation faults.

A key technique for coping with faults in a computing system, redundancy has been established~\cite{TanenbaumSteen07}.
In redundant computation, safety-critical calculations are performed in parallel in a distributed manner and combined afterwards.
Generally speaking, redundancy entails additional resources, that are not required for a system's functional operation, but adds certain characteristics such as fault detection and fault tolerance~\cite{BarryFaultToleranceAnalysis}.
Various well-established redundancy patterns have been proven in practice, which include hardware-, software-, information-, and time redundancy.
In contrast to non-redundant systems, redundancy allows to build safe systems out of less safe and cheaper parts.
\\

For handling communication in distributed systems, different concepts, standards and frameworks exists.
One of them is \abr{DDS}, a \abr{DCPS} standard for machine-to-machine communication that is specified by the \abr{OMG} and intended to be implemented as a middleware~\cite{omgDDSspec}.
\abr{DDS} is designed for facilitating safe and real-time communication among distributed and autonomous execution units and is therefore applicable for building safety-critical systems~\cite{DistributedSafety2020}.
The standard, which is intended to be implemented as a middleware, allows the specification of \abr{QOS} to specify the service's behavior in a declarative way.
Through the \abrpl{QOS}, the communication can be defined to be reliable and real-time.

However, \abr{DDS} has been designed to be a generic standard for distributed application communication and integration and therefore provides a wide variety of features~\cite{omgDDSspec}.
Hence, when applied in a safety-critical application, every feature from the applied middleware implementation needs to be approved.
\\

In this work, it is investigated whether the \abr{DDS} standard is applicable for building redundant, distributed, real-time, and safety-critical systems in the \abr{ETCS} domain.
For this purpose, different well-established redundancy concepts are analyzed towards their safety and feasility to be implemented with \abr{DDS}.

First, an overview about popular redundany patterns and characteristics for building safe and reliable systems, as well as about \abr{ETCS} and \abr{DDS}, is given in~\autoref{chptr:concepts}.
Afterwards, mathematical concepts for evaluating the safety and reliability of these systems are pointed out.
A selection of related work that uses \gls{DDS} in safety critical applications, is presented and set in relation to this work at the end of~\autoref{chptr:concepts}.
\\

In~\autoref{chptr:redundantSystemsCompare}, key building blocks for designing distributed systems with the \abr{DDS} standard are described.
The redundancy concepts and patterns that were previously presented are opposed and evaluated based on the characteristics from~\autoref{chptr:concepts}.
Further, exemplary concepts and architectures for each pattern are described and assessed based on their safety and realizability with \abr{DDS} concepts.
Therefore, not only voting-based concepts, but also a way of building consensus driven redundant systems, are described and analyzed.
A combination of a consensus- and voting-based approach is identified as the most suitable concept for satisfying the desired system's requirements.
\\

In~\autoref{cpt:Implementation}, an exemplary implementation is presented that follows the concepts of \texttt{Raft}~\cite{RaftConsensusPaper}.
Thereby, the significance of a safe consensus- and voting-based redundant system, which utilizes \abr{DDS} for communication, is shown.
The system comprises of four homogeneous and interconnected \textit{Revolution Pi} from Kunbus~\cite{Kunbus} and utilizes \textit{OpenSplice DDS} from ADLINK~\cite{VortexOpenSplice} as a \abr{DDS} framework.
A \textit{Revolution Pi} is a robust \abrpl{PLC} that is build upon a \textit{Raspberry Pi} and features a real-time capable operation system.
The \textit{OpenSplice DDS} framework's applicability for safety-critical railway applications has been proven in practice~\cite{SchmidtMissionCriticalChallenges}.
\\

Afterwards, the implemented system's safety and functional correctness is proven based on a simulated \abr{ETCS} scenario in~\autoref{cpt:evaluation}.
Results show the the system runs the simulated scenarios successfully, even when an entire component fails.
In addition, the system's time performances and hardware resource utilization is examined.


%TODO Mention results and main realization


    \chapter{Background and Related Work}
\todo{Explain somewhere that redundant communication channels might be a good idea}
\label{chptr:concepts}

Introducing redundancy to safety-critical systems is a popular approach to ensure that faults inside the system do not lead to failures that effect the system's environment~\cite{BarryFaultToleranceAnalysis}.
Redundancy can be achieved in multiple ways, some of which being adding additional resources (hardware redundancy), adding additional information to a message (information redundancy) or performing the same operation multiple times (time redundancy).
Although being one of the most often used redundancy techniques, the addition of hardware components, called replicas for redundant computation, leads to multiple results which need to be consolidated to an individual system result.
Thus, the replicas need to communicate with each other and synchronize their results in order to act like a single unit in their environment.
A promising approach to solve the communication overhead could be \gls*{DDS}, a \gls*{DCPS} system, since it allows reliable and real-time communication~\cite{omgDDSspec}.
However, while hardware redundancy enhances a system's fault-tolerance, it also increases its costs and complexity.

For weighting out different redundancy concepts, various redundancy techniques, important terms and criteria for evaluating redundant systems are described in this chapter.
Afterwards, standards for building and designing safe and fault-tolerant systems are given and related works, that sucessfully use \gls*{DCPS}-middleware in the railway context, are shown.
At first, however, an overview about the \gls*{DDS} standard and its concepts are exposed.

\section{Data Distribution Service}

\begin{figure}[!hb]
	\centering
	\includegraphics[width=0.75\linewidth]{images/DDSStructure}
	\caption{\abr{DDS} is as data-centric communication model and follows the publish/subscribe pattern. The publishing and subscribing attendees do not communicate directly with each other, but make use of topics to read and write data objects.}
	\label{fig:DDSStructure}
\end{figure}

\Gls*{DDS} is a \gls*{DCPS} model for machine-to-machine communication, that is specified by the \gls*{OMG}~\cite{omgDDSspec}.
It is stated that the model should, for practical use, be implemented as a middleware, to directly interface with the underlaying \gls*{OS}.
The model follows the concept of a global data space to facilitate data exchange among entities based on type and content.
Central entities in the design are \texttt{Publishers} and \texttt{Subscribers}, while data exchange is based on \texttt{Topics}.
A \texttt{Publisher} is used for sending data of different types.
In order to send specifically typed data, a \texttt{DataWriter} object is used, which acts as a typed interface to a \texttt{Publisher}.
On the other side, a \texttt{Subscriber} is used to receive data and to make it accessible to the receiving application.
The equivalent of the \texttt{DataWriter} for the \texttt{Subscriber} is the \texttt{DataReader} object.
A \texttt{Topic} acts as the connecting element between publishing entities on the one side and subscribing entities on the other.
Any forecasting of when and how data is published to a \texttt{Topic} or received by a \texttt{Subscriber} is made possible through so called \glspl*{QOS} policies.
This concept is illustrated in~\autoref{fig:DDSStructure}, which is based on the official \gls*{DDS} specification~\cite{omgDDSspec} and shows the information flow from the publishing to the subscribing side.

The usefulness of \gls*{DDS} for containerization technologies in automotive architectures has been studied by Kugele \etal~\cite{KugeleDataCentricForAuto}.
Although their findings are based on containerization technologies and are therefore not resilient for this work, Kugele \etal showed that the applicability of \gls*{DDS} with regards to safety, certification and security depends on the actual implementation's characteristics~\cite{KugeleDataCentricForAuto}.
Therefore, Vortex OpenSplice DDS is used thoughout this thesis, which is being developed and maintained by ADLINK.

The applicability of the \gls*{DDS}-standard and of Vortex OpenSplice \gls*{DDS} for safety-critical railway applications has been demonstrated by Schmidt and van't Hag~\cite{SchmidtMissionCriticalChallenges}.
Their findings show, that OpenSplice \gls*{DDS}'s \gls*{QOS} policies allow predictable time and locality characteristics for data distribution.
Further, their work provides an overview about Vortex OpenSplice's \gls*{QOS} policies and features to ensure reliability, availability, data-delivery, and resource usage.

\section{Techniques for Safety and Reliability Evaluation}
\label{sec:techniquesSafetyReliability}
In order to profoundly express and analyze redundancy techniques towards their safety, reliability and fault-tolerance, it needs to be defined what these characteristics mean.
\\

Reliability is, by IEEE 610.12-1990, defined as the ability of a system or component to perform its required function under stated conditions for a specified period of time~\cite{ieee610.12}.
A circumstance where a system deviates from its requirement is called a system failure.
A failure is preceded by a fault, which describes a static defect of a system~\cite{AmmannOffutt2016}.
When a fault becomes active, it manifests itself as an error, which marks an incorrect internal system state.
An failure occurs as soon as the error affects the system's environment.

\begin{definition}
A failure of a system or a system component is a state, where its actual behaviour deviates from the its specified behaviour.
\end{definition}

The definition of reliability used in this work is the following:

\begin{definition}
A system's reliability is a function of time that expresses the probability for the system to operate as specified at a time $t_1$, given that it was operating as specified at time $t | t \leq t_1$.
\end{definition}

In other words, reliability is a system's ability to not have any failure for a specific period of time.

A system's safety is associated with its reliability, since it consitutes an extension of reliability~\cite{AvizienisDependability2001}.
In general, safety defines the absence of catastrophic failures on a system's environment.
When the presence of non-catastropic failures can be reliably detected and a safe state can be taken in case of failures, safety can be treated as reliability concerning catastropic failures.
In other words, safety can be expressed as a system's probability to not experience any fault that would lead to a catastrophic failure in a specific timespan.

Based on these definitions, it can be conducted that a system's safety and reliability can only be finally assessed when the system's environment, the reliability of the system's components and the operations performed by the system are known.
In addition, the system's architecture and structure needs to be aquainted.
Finally, in order to specifically evaluate a system's safety, possible faults and their consequences need to be known.
In order to be able to evaluate and compare different architectural pattern towards their safety characteristics, independently of the subsequently conducted operations, the definition of \texttt{intrinsic safety}, made by~\cite{BoulangerStandards}, is used in this work.

\begin{definition}
A system is said to be intrinsically safe if one can be certain that any failure of one or more components of that system will only result in its becoming more permissive.
In the railway context, a complete stoppage is generally the safest state.
\label{def:intrinsic_safety}
\end{definition}

This definition requires that measures are taken to assure the system's safety even in the presence of failures.
A system that provides the functionality to behave as specified even in the case of faults is said to be fault tolerant~\cite{AvizienisDependability2001}.
Fault tolerance is generally obtained through error detection, which allows the system to determine the presence of errors and enables it to mitigate resulting failures.
In the course of this thesis, a system is considered to be safe when it is reliable and utalizes error detection mechanisms to ensure its intrinsic safety.
\begin{definition}
A system, in the course of this thesis, is said to be safe when it operates reliably and applies error detection mechanisms in order to be fault tolerant.
\label{def:safety}
\end{definition}

Thus, in order to be considered secure, a distributed system must tolerate possible faults that could occur in distributed systems.
Flaviu Cristian has established the following five fault classes for distributed distributed systems~\cite{CristianFaultModel}:

\begin{enumerate}
\item \textbf{F1:} One or multiple components in the system crash (\textbf{crash fault}).
\item \textbf{F2:} One or multiple components fail to respond to an incoming request (\textbf{omission fault}).
\item \textbf{F3:} One or multiple components fail to produce an output within a certain time span (\textbf{timing fault}).
\item \textbf{F4:} One or multiple components produce a wrong result (\textbf{computation fault}).
\item \textbf{F5:} One or multiple components produce arbitrary responses at arbitrary times (\textbf{byzantine fault}).
\end{enumerate}

It applies, that all failure classes \textbf{Fx} are included in \textbf{Fy}, given that $y \geq x$.
In the course of this thesis, the fault classes \textbf{F1} to \textbf{F4} are analyzed and byzantine faults are not covered.

\subsection{Safety Evaluation}
\label{sec:safetyEvaluation}
A required condition for a system to be safe is that the system is reliable~\autoref{def:safety}.
One method of analyzing a system's reliability is through mathematical functions of time, for example the \texttt{exponential failure law}~\cite{GeffroyMotetDependableComputing}.
It describes a component's reliability as an exponential function on time.
\begin{equation}
R(t) = e^{-\lambda t}.
\label{eq:expFailureLaw}
\end{equation}
The parameter $\lambda$ is called the failure rate and encodes the probability of failures occuring in a certain time span, typically in an hour.
For the exponential failure law it is, mathematically, assumed that the components fail independently.
This might not always be the case in practice, as common core or power outage failures can affect multiple components at once.
However, the probability of failures affecting multiple components at once can be reduced by various techniques, such as using independent power supplies or diverse redundancy, the latter of which being discussed below.
\\

Among various theoretical techniques for evaluating a system's safety, Markov chains are one of the most commonly used~\cite{BarryFaultToleranceAnalysis}.
Markov chains are a stochastic process that models the alterations of a system's state and probabilities for the system to transition into a specific state given that it was in a certain state~\cite{KemenyMarkovChains}.
Thereby, the next state only depends on the current state and is independent from all previous system modifications.

For a safety analysis, each state models the system in one out of three conditions:
\begin{enumerate}
\item The system functions without having an error.
\item The system detects and successfully mitigates an error without leading to a failure
\item The system either not detects an error or does not recover from it, leading to a failure
\end{enumerate}

For the state transition probabilities, the exponential failure law~\autoref{eq:expFailureLaw}, with a constant failure rate $\lambda$, holds.

\begin{figure}[!hb]
	\centering
	\includegraphics[width=0.75\linewidth]{images/TriplexSystemNASA}
	\caption{A reliability model for \gls*{TMR} using Markov chains. In state (1), three replicas are operating redundantly. Each state transition describes the failure of a replica together with the corresponding probability.}
	\label{fig:NASATMR}
\end{figure}

An example Markov chain for a system using three replicas is proposed by NASA and depicted in~\autoref{fig:NASATMR}~\cite{NASAMarkovChains}.
At state (1), assuming that the system is homogeneous redundant, the probability that one of the replicas fails is $3\lambda$.
In state (2), the system applies failure detection and mitigation procedures and has a chance of $F(t)$ for successful reconfiguration, which allows the system to continue its work.
However, there is still a change of $2\lambda$ that another component fails before the system detects and mitigate the first failure, which would lead to a system fault and render the entire system unsafe.
In state (4), there is again a $2\lambda$ change for one of the remaining components to fail, which would lead to a failure of the entire system.

As experiments have shown, a system's recovery time is not necessarily exponential~\cite{TheoryAndPracticeReliableSystem}.
Thus, it is expressed by $F(t)$ which is the probability that the system recovers within a timespan less than $t$.

In a diverse redundant system, each component failure is represented with an individual state.

\section{Redundancy Patterns}
\label{sec:redundancyPatterns}
As already stated, a typically applied technique for handling and masking errors in a system is redundancy~\cite{TanenbaumSteen07}.
Error masking is the concept of detecting and mitigating errors, so that they do not become failures and effect the system's environment.
For redundancy, additional resources or information are added to a system, that would not be required when errors where impossible to happen in the system.
Barry Johnson defines redundancy in the following way~\cite{BarryFaultToleranceAnalysis}:
\begin{definition}
The concept of redundancy implies the addition of information, resources, or time beyound what is needed for normal system operation.
Redundancy can take one of several forms, including hardware, software, information, and time redundancy.
\end{definition}

Each form of redundancy has its unique characteristics and patterns, which are presented in the following.

\subsection{Hardware Redundancy}
In hardware redundancy, additional replications of physical components are added to the system.
This typically increases the system's safety by masking internal failures, but also increases the system's cost.
In general, a system's safety can be further improved when using components that are based on different internal components.
Using different components in redundancy patterns is called diverse redundancy, while replicating the same components is called homogeneous redundancy~\cite{HomogeneousRedundancyOuzineb}.
Using diverse redundancy has the benefit that it reduced the effect that common core errors have on a system's or component's safety.

Johnson subdevides hardware redundancy into two parts, namely passive and active redundancy.
In passive hardware redundancy, a voter or consensus algorithm is used to reduce a number of redundant outputs to a single output, in order to prevent individual internal failures from propergating out of the system.
In active hardware redundancy, the system tries to detect and repair any internal failure, for example by replacing the faulty component.
While \gls*{MOON} systems are a typical example of passive hardware redundancy, standby redundancy is often used as an example of active hardware redundancy.

A voter can be realized in software and in hardware.
The benefit of software voters is, that they typically cost less and are easier to develop, because no special hardware is required.
However, software voters operate slower and the approval process is more complicated, because additional hardware and software is used, that does not contribute to the voting.

The benefit of hardware voters is, that they can operate faster and a minimal set of hardware needs to be approved, which reduces the time and cost expenses.
Both types of voters have in common, that they require some kind of synchronization with the replicas to recognize failed or delayed redundant results.

\paragraph{M-out-of-N Systems}
\begin{figure}[!hb]
	\centering
	\includegraphics[width=0.75\linewidth]{images/Classical2OO3}
	\caption{Classical 3-out-of-2 redundancy, also known as \Gls*{TMR}. Three replicas are simultaneously reading and processing an input in a redundant way. A voter collects these redundant results and performs a majority voting to produce a final output.}
	\label{fig:Classical2OO3}
\end{figure}

One of the most common versions of \gls*{MOON} systems is \gls*{TMR}~\autoref{fig:Classical2OO3}.
In \gls*{TMR}, three replicas are receiving the same input and performing the same work in parallel.
A voter collects the individual outputs and reduces them to a single system output based on a majority vote.
The system output must qualify the characteristic that at least two out of the three replicas agree on this output.
This allows the entire system to still produce a correct output even in the presents of a faulty replica.
All replicas, as well as the voter, are running on individual hardware units.

The weakness of \gls*{TMR} is the voter, because it marks a single point of failure.
Therefore, the voter's reliability in this system has to be very high compared to the three replicas, because the entire system's dependability cannot be higher than the voter's dependability.
There is a lot of research for making voters more reliable, one of which being Arifeen \etal, who proposed a highly reliable hardware voter~\cite{ArifeenFaultTolerantTMR}.
As an alternative to the voter, a consensus algorithm could be used.
A consensus algorithm is applied to let all components in the system agree on a single output value.
This single output value is required to be proposed by at least one component~\cite{lamport2001paxos}.
While not having a single point of failure anymore, consensus approaches typically introduce a communication overhead and lead to rigid configurations~\cite{GamerIncreasingMOON}.

\begin{figure}[!hb]
	\centering
	\includegraphics[width=0.75\linewidth]{images/IntermediateVoting}
	\caption{Intermediate voting could be applied to reduce the effect of partial results on the final output. Another benefit of this approach is its pipelined fashion, which improves the overall system throughput.}
	\label{fig:IntermediateVoting}
\end{figure}

Instead of having one voting about the final result, the calculations could be modularized and votings on intermediate results could be made.
This would have the benefit of masking intermediate faults and not letting them influence further calculations.
An example of how this could be achieved is depicted in~\autoref{fig:IntermediateVoting}.

\paragraph{Standby Redundancy}
\begin{figure}[!hb]
	\centering
	\includegraphics[width=0.75\linewidth]{images/ActiveSelectionRedundancy}
	\caption{With standby redundancy, a system can recover from individual component errors by replacing faulty components. Therefore, a switcher component performs error detection operations and delegates control over the output to the component it considers to be the most trustworthy.}
	\label{fig:StandbyRedundancy}
\end{figure}

As stated above, standby redundancy is an example of active hardware redundancy and builds on the concept of error detection, location and recovery.
In an exemplary case of an internal component failure, the resulting error needs to be detected and the faulty component needs to be located, so that the system can recover from the error by replacing the faulty component by an identical spare component.
An example is depicted in~\autoref{fig:StandbyRedundancy}, where two identical replicas are used, one as a primary and one as a secondary component.
A switching element observes the active component and, when no internal failure in the active component happened, leads the active components output out of the system.
As soon as the switching element detects an internal failure, it excludes the active component from the system and takes the secondary component's output as the system's output, which thereby becomes the active component.
When the secondary component is already running when the switching happens, this is called hot standby redundancy.
When the secondary component needs to be turned on before it can be used as the active component, this is called cold standby redundancy.

\subsection{Software Redundancy}
One speaks of software redundancy, when redundancy concepts for error detection are implemented in software.
Examples for software redundancy are heartbeat messages to validate a component's accessibility, component-checking or software voters.
In component-checking, a program is used to monitor and validate a component's hardware, for example its memory, clock or \gls*{CPU}.
Thereby, it can be qualified whether a specific component's output can be considered valid or not.
This state is summarized into the term \texttt{internal consistency}.

\begin{definition}
A system or a component is said to be internally consistent, when any failure of this component is not based on a fault of its internal hardware.
A system's or component's internal consitency can be determined by component checking mechanisms.
\end{definition}

Further, voting or consensus algorithms can be implemented in software.
Similar to homogeneous and diverse hardware redundancy, diverse implementations of the same software can be used to reduce the effect of faults in a software.
The concept of diverse software programs is summarized under the term N-version programming, where a software program, which is based on the same specification, is developed by separate programmers~\cite{BarryFaultToleranceAnalysis}.

\subsection{Information and Time Redundancy}
The addition, adding redundant information can allow the detection, masking and recovery from faults~\cite{BarryFaultToleranceAnalysis}.
Examples for information redundancy include Hamming codes~\cite{HammingCodes}, checksums or information duplication.
All information redundancy concepts share that they somehow encode redundancy into some information that can later be decoded.

\begin{figure}[!hb]
	\centering
	\includegraphics[width=0.75\linewidth]{images/ECC}
	\caption{Error checking and correction applies both time and information redundancy. An encoder is used to encode both input and output of a system (time redundancy). This encoded information is used in addition to the actual data (information redundancy).- The encoder function needs to be choosen to that possible outputs can be derived from it~\cite{Su2005ECC}.}
	\label{fig:ECC}
\end{figure}

Another way of adding redundancy is to perform the same operation multiple times at different points in time.
This concept is called \texttt{Time Redundancy} and has the benefit that it does not require many additional physical components.
Thereby, permanent faults in a system can be detected.

\autoref{fig:ECC} depicts an example of a combination of information and time redundancy using \gls*{ECC} techniques.
Therefore, an encoder encodes potential outputs for certain inputs and transmits this redundant information together with the input data.
After the system produces an output based on its input information, another encoder encodes the output, which is afterwards compared to the encoded input.
When there are anomalies between the system's input and its output, safety operations can be initiated.
The encoding function must be choosen so that it allows the detection of faults in the system, such as arithmetic shifting.

\section{Related Work}

Alapan Chakroborty demonstrates a development process of a reliable, safe and fault-tolerant system using redundancy methods by an example of railway signalling~\cite{ChakrabortyFaultTolerantRailway}.
He argues, that every fault tolerant system needs to build upon real-time software solutions.
This requires both logical, as well as timing correctness, which can only be achieved by correct redundancy management such as fault propagation, synchronization and consensus among replicas.
At a fail-safe design's heart lies fault idenfication, masking and recovery.
Therefore, in a first step, each redundant design should be subdevided into redundant elements that are not affected by any fault from outside the element.
Chakroborty calls these redundant elements \glspl*{FCR}.
As a second step, he proposes to define interface between the \glspl*{FCR} so that they do not interfere with each other.
These two steps form a necessary precondition to be able to predict the probability of failures in the system.

In order for the system to reduce redundant results to a single result and to mask failures, the concept of voting is demanded.
Voting, as Chakraborty claims, requires the \glspl*{FCR} to be in identical states, to provide all redundant hardware with the same input and to ensure a real-time concurrent operation of the same computations on each \gls*{FCR} for synchronization.
Finally in the work, the shown method is performed on developing a railway signalling system.

An overview about possible redundancy concepts for the railway domain is provided by Bemment \etal~\cite{BemmentEvaluationOfRedundancy}.
They further made a cost, benefit and performance analysis for each presented concept.
\\

In my work, the concepts proposed by Bemment \etal are extended based on the steps by Chakraborty and by using and evaluating the feasibility of \gls*{DDS} for redundancy in the railway context.
Normally, as Bemment \etal pointed out, an in depth redundancy evaluation towards safety and reliability can only be made for well defined use cases where hazards, risks and potential accidents are know.
In my work, however, only the intrinsic safety in examined, which can be done theoretically without doing an exact risk analysis.
For further evaluations, an overview about about possible accidents in railway is given by~\cite{ERTMSRailwayAccidents}.
\\

The CENELEC 50128 standard is a norm for software creating processes for railway applications, so that the build software can be considered safe.
The 2011 version of CENELEC 50128, as well as resources needed to achieve a set level of assurance, are presented by Jean-Louis Boulanger~\cite{BoulangerStandards}.
Conducting a CENELEC 50128 conform design process for all software artefacts created in this thesis is outside of its scope.
However, each software indended to be used in railway practice should pass the CENELEC 50128 norm.
\\

\todo{Adapt table so that I extend the QOS from the paper with those that I need. Also drop those from the paper that I do not need}
\begin{table}[h!]
	\begin{center}
		\caption{\Gls*{QOS} policies that affect the communication overhead (o) or the communication time (t). Each \abr{QOS} policy can either be applied to \texttt{DataWriters} (DR), \texttt{DataReaders} (DR), \texttt{Topics} (T), or \texttt{Publishers} (P).}
		\label{tab:qos_garciavalls}
		\begin{tabularx}{\textwidth}{|l|l|X|}
			\hline
			\textbf{QoS policy} & \textbf{Entity} & \textbf{Description}\\
			\hline \hline
			Deadline (t) & DR \& DW & Max exprected elapsed time between arriving data samples or instances. Max committed time to publish samples or instances.\\
			\hline
			Reliability (o) & DR \& DW & Global policy that specifies whether or not data will be delivered reliably. It can be configured on a per DataWriter/-DataReader connection. \\
			\hline
			History (o) & DR \& DW & Stores sent or received data in cache. It affects the Reliability \gls*{QOS} policy. \\
			\hline
			Resource Limits (o) & DP & Limit to the allocated memory. It limits the queue size for History when the Reliability protocol is used. \\
			\hline
			Latency Budget (t) & T \& DR \& DW & Indication on how to handle data that requires low latency. Allow specification of maximum acceptable delay from time the data is written to the time the data is received by the subscriber. \\
			\hline
			Time based Filter (t) & DR & Limits the number of data samples sent for each instance per a given time period. \\
			\hline
			Transport Priority (t) & DW & Establishes a given priority for the data sent by a writer.\\
			\hline
		\end{tabularx}
	\end{center}
\end{table}

The \gls*{DDS} is already successfully deployed in distributed time- and safety critical environments.
One example is given by Bijlsma \etal~\cite{DistributedSafety2020}, who extended the E-Gas layered monitoring concept to handle faults in a distributed and redundant system.
\Gls*{DDS} is applied to facilitate a reliable communication among individual components in the system.
An other example is given by Hadiwardoyo and Gao, who proposed the use of \gls*{DDS} in security cameras for subways using \glspl*{QOS}~\cite{DDSInSubways}.
A more general approach of how \gls*{DDS} can be used for enabling communication in distributed as well as time- and safety-critical environments, is given by García-Valls \etal~\cite{GarciaVallsDDSInDistributed}.
They used a design of reading and monitoring sensor data to benchmark the middleware's communication performance with different \glspl*{QOS}.
One of their findings is a summary of \gls*{QOS}-policies that effect the system's communication overhead and communication time.
These are illustrated in~\autoref{tab:qos_garciavalls} and are especially important when examining real-time and mission-critical systems.
The results from García-Valls \etal show that the communication performance remains stable even under heavy load.
Even though they applied a different \gls*{DDS} implementation than in my work, the network topology and used hardware are comparable, which renders the results from García-Valls \etal useful for my work.

    \chapter{Concept Development}

\iffalse

In this chapter, typical system architectures for solving highly safety-critical tasks are presented and discussed.

One way of building a dependable system, even out of undependable components, is through distribution (Source ReliabilityEngineering Slides).
This is because the failure of one individual component in a distributed system does not affect the remaining component's dependability.
However, if the failing component was responsible for a crucial system function, then a single failure can lead to a failure of the entire system.
A commonly applied technique for maintaining a system's functionality even when a crucial componant fails is redundancy~\cite{TanenbaumSteen07}.
Therefore, the crucial component's work is performed on multiple components, called replicas, simultaneously.
When one replica fails, the remaining replicas still allow the system to continue its work.

In the following of this chapter, different distributed and redundant system architecture approaches are described and assessed.
At first, redundancy categories that have been proven both in practice and in theory are described, based on~\cite{GeffroyMotetDependableComputing} and~\cite{BarryFaultToleranceAnalysis}.
Afterwards, the most prominent representatives out of the presented redundancy categories are choosen and evaluated based on the requirements made in~\todo{Ref to chapter 1}.
Finally, a decision for the most suitable system architecture for the described use case is compiled.

\section{Redundancy Techniques}


\subsection{Redundancy}

However, adding redundancy also increases the complexity as well as design and building costs of a system.

In computing systems, the two types \texttt{functional}- and \texttt{structural}-redundancy can be found.
While functional-redundancy incorporates a system's external manner, structural-redundancy refers to its internals.

\subsubsection{Functional Redundancy}
Functional redundancy is encoded in the relationship between a system's input and its output.
It allows the system to filter out impossible or unacceptable inputs or outputs.
Functional redundancy is independent of the system's design and implementation and solely depends on its functions.
A way of adding functional redundancy to a system is to structure it into interconnected modules~\cite{GeffroyMotetDependableComputing}.
\todo{Explain this further}

\subsubsection{Structural Redundancy}
For structural redundancy, additional components are added to the system that would be unnecessary, provided all components are working correctly.
Structural redundancy can be applied in software, in hardware or in the time domain~\cite{GeffroyMotetDependableComputing}.
Examples for structural redundancy are:

\paragraph{Standby Redundancy}
In standby redundancy, the redundant components are subdevided into primary and secondary components.
A third component type is required to switch between the primary and secondary components, when required.
The switching step, however, is not possible without any costs and adds an additional point of failure~\cite{PepperlFuchs}.
Standby redundancy is subdevided into cold and hot standby, implicated whether the secondary components are turned off or on while they are not used.

\paragraph{M-out-of-N Redundancy}
Another popular redundancy pattern for industrial applications is \gls*{MOON}-redundancy~\cite{GamerIncreasingMOON}.
Thereby, N individual components are executing the same task simultaneously and all separate outputs are collected.
The system finally generates the overall output by selecting the value where at least M out of the N components agreed on.
This allows the system to generate the right output even if $N-M$ components produced a wrong output.



\todo{Define module composition}

In structural redundant systems, it often happens that multiple redundant outputs need to be reduced to a single system output.
In order to produce a single output out of $N$ individual outputs, a voter or a consensus algorithm could be used.

\paragraph{Voter} 
A voter is an entity that collects the components individual outputs and performs a voting algorithm to combine them to a single output.
The voter approach has the benefit that it is fast and easy to implement.
For voter based systems, Chow and Willsky argue that system output's residual generation can be done by simply comparing outputs of redundantly working components~\cite{ChowFailureDetectionSystems}.
However, a voter is a single point of failure and thereby impairs the entire systems reliability and availability.

\paragraph{Consensus} 
A consensus algorithm is applied to reach consensus among the system's individual components in order to agree on a single output value.
This single output value is required to be proposed by at least one component~\cite{lamport2001paxos}.
While not having a single point of failure anymore, consensus approaches typically introduce a communication overhead and lead to rigid configurations~\cite{GamerIncreasingMOON}.


\section{Choosing a System Architecture}







\autoref{fig:Classical2OO3} depicts the most prominent form of passive hardware redundancy, \gls*{TMR}~\cite{BarryFaultToleranceAnalysis}.


The weakness of \gls*{TMR} is the voter, because it marks a single point of failure.
Therefore, the voter's reliability in this system has to be very high compared to the three replicas, because the entire system's dependability cannot be higher than the voter's dependability.
A solution to this problem would be to increase the number of voters.
However, having multiple voters would also lead to having multiple outputs and would require an additional step to combine the individual outputs to a final output, being a new single point of failure.

It is assumed that all replicas follow the exponential failure law.
Based on \autoref{eq:expFailureLaw} and \autoref{eq:reliabilityMOON} follows for the replicas reliability
\begin{equation}
R_{2oo3}(t) = \sum_{i = 2}^3 {3 \choose i} * (e^{-\lambda t})^i * (1 - e^{-\lambda t})^{3 - i}
\end{equation}

%
%
%


\section{Error Checking and Correction}



%Throughput can be increased by combining the parallel and serial pattern in a pipelined way.
%This also has the way that it adds functional redundancy to the system~\cite{GeffroyMotetDependableComputing}.


\todo{Define several candidate designs}
\todo{Analyze candidate designs}


\fi

    \chapter{Implementation}
\label{cpt:Implementation}

In this chapter, a practical realization is presented which implements a redundant architecture that applies a consensus algorithm based on the concepts of \texttt{Raft}.
The aim of this implementation is to showcase the practicability and performance of a redundant architecture that applies \abr{DCPS} concepts for finding a consensus in an \abr{ETCS} context, even in the presence of single component failures.
In addition, a minimal required subset of \abr{DDS} features will be carved out to implement the architecture.
This has the potential to safe costs during implementation and system approval.
\\

\begin{figure}[!hb]
	\centering
	\includegraphics[width=0.7\linewidth]{images/setup}
	\caption{Three \textit{Revolution Pis Connects} make up the redundant system which is implemented in the course of this thesis. The replicas are interconnected via a network switch. A fourth \textit{Revolution Pis Connect} acts as a spare unit and can be activated if needed.}
	\label{fig:SystemSetup}
\end{figure}

The system setup is depicted in~\autoref{fig:SystemSetup}, where four replicas are interconnected via a network switch.
Because of the network switch being a single point of failure in the applied star network, a fully connected network would be the better choice.
A star network does not change the functionality of the setup and was therefore chosen for the demonstration for cost reasons.

While three replicas are operating in a \abr{TMR} setup, the fourth is used as a spare unit that can be patched in on demand, for example when an other replica failed.

Each replica is represented by a \textit{Revolution Pi Connect}, an open source \abr{PLC} that builds upon a Raspberry Pi and is developed by \texttt{Kunbus}.
As such, it features a Broadcom BCM2837 quad core ARM Cortex A53 1.2 GHz \abr{CPU} and 1 GB RAM.
Further, a hardware watchdog is applied that can be configured to react when the program stops responding.
For implementing this, a watchdog timer is deployed that needs to be manually resetted by an application.
When this is not done for a certain period of time, the system assumes that the application crashed and the system is restarted.
A watchdog is an important safety-feature and proposed by Sakic and Kellerer~\cite{SakicTimeInConsensus} to increase a system's probability to response when a component failed.
The used \abr{OS} is an extended \textit{Raspbian} \abr{OS} that has been patched with real-time support.
The solution is implemented in the C programming language and makes use of the C \abr{API} of \textit{Votex OpenSplice DDS} by \texttt{ADLINK}.

The communication channel is implemented via \textit{Ethernet} and no information redundant communication channel is applied in this demonstration.
Further, no N-version programming is applied, which means that each replica runs the same software implementation.
In addition, it is assumed that all replicas function as intended, so that component checking mechanisms are neglected in this implementation.
In order to build a highly secure system, the mentioned compromises need to be further addressed.

\begin{figure}[!hb]
	\centering
	\includegraphics[width=0.9\linewidth]{images/Components}
	\caption{The on-board unit's program, that runs on each replica, consists of three major components. The \texttt{input} processing reads inputs and performs the voting. While the \texttt{state manager} components administers the system's global state, the \texttt{consensus} component ensures that a leader is present in the system and provides an interface to process an input on the cluster and collect all results.}
	\label{fig:SystemComponents}
\end{figure}

A general outline of the system is structure is provided in~\autoref{fig:SystemComponents}.
The system's \texttt{Input Processing} component provides an interface to channel balise or \abr{RBC} telegram messages as inputs and an output where the final result for a corresponding input is presented.
The final result is generated by invoking the \texttt{Consensus} component's \textit{cluster\_process} interface.
Thereby, the input is distributed across the cluster using \abr{DDS} topics and intermediate results are collected.
When enough intermediate results are generated within a certain time-span, \textit{on\_result} is called, otherwise \textit{on\_fail} would be invoked so that the \texttt{Input Processing} component could initiate appropriate consequences.
The \texttt{Consensus} component manages the leader election process, as well as a private replica state.
However, every other decision, that is independent of the consensus algorithm, is made based on a global system state.
This global state is managed by a corresponding \texttt{State Manager} component.

The individual components are described in more detail later in this chapter.
At first in this chapter in~\autoref{sec:ImpScenarioDescription}, a subset of \abr{ETCS} is mapped out that serves as the practical scenario for the implemented system.
Afterwards,~\autoref{sec:ImpConsensusAlgorithm} describes a consensus algorithm, as part of the \texttt{Consensus} component, that builds on \abr{DDS} and ensure that the redundant system can agree on a certain result for each processed input telegram.
Therefore, a leader election process is described, which follows the concepts of \texttt{Raft}, and timing requirements will be taken into account to ensure safety in the redundant system.
The leader election process lays the foundation for the way of how input telegrams are spread across the system and merged into a single final value, which is further described in~\autoref{subsec:ImpBaliseProcessing}.
Thereafter, is is described how the \texttt{State Manager} component manages the system's state and how the state can be updated and queried.
Finally in this chapter, an approach for implementing a simulator is described before a hot standby scenario with \abr{DDS} features is presented.

\section{Scenario Description}
\label{sec:ImpScenarioDescription}
The exemplary implementation, that is described in the following of this chapter, shall comply with \abr{ETCS} level two and receive, evaluate, as well as react to certain \abr{RBC} and track-side balise telegram inputs.
In \abr{ETCS} level two, track-side balises are solely used for detecting and correcting the train's exact position.
As stated in the \abr{ETCS} specification, track-side balises are organized in groups and the combination of all telegrams from balises that belong to a group form a balise group telegram~\cite{ETCS26}.
Each balise group contains at least one and maximal eight balises.
Besides telegrams, each balise group is assigned with a coordinate system that encodes its orientation, and a position.
Therefore, track-side balises can be used both for transmitting information to a on-board unit and for correcting the confidence interval, which means adjusting a trains calculated position.
In order to do so, the expected position of each balise group, which the operating train will encounter on its route, is communicated pre-journey during a linking step.
When a train does not locate a balise group where it is expected, or registers an unlinked balise group, appropriate actions must be taken to ensure safety.

Another important information for a safe train operation is a \abr{MA}, which indicates a location up to which a train is authorized to move, as well as a speed profile and a gradient profile.
In order to ensure compliance with a \abr{MA}, the on-board unit must continuously monitor the train's speed, position, and breaking curve so that it does not disregard the track's profiles or pass the \abr{MA}'s end location.
In \abr{ETCS} level two, \abrpl{MA} with speed and route information are continuously transmitted to the train via wireless communication technologies, such as \abr{GSMR}, from the \abr{RBC}.

It is the on-board unit's responsibility to supervise the train's position, speed, movement authorities and braking curve, as well as to evaluate balise telegrams and \abr{RBC} messages.

An algorithm for calculating a train's braking curve in \abr{ERTMS} systems is presented by B. Friman~\cite{CalculateBrakeCurveFriman}.
In general, three type of information are considered for supervising a train, namely speed restrictions, track gradients and a train's characteristic deceleration abilities.
For calculating a train's braking distance $dbr$ for a train with a deceleration ability of $dec$ from one speed $v_1$ to another speed $v_2$, the following equation can be used:

\begin{equation}
dbr = \frac{{v_1}^2}{2*dec} - \frac{{v_2}^2}{2*dec}
\end{equation}

The train's speed is measured in $m/2$ and its deceleration ability is measured in $m/s^2$.
\\

In order to show the demonstrated redundant system's applicability, a subset of the on-board unit's \abr{ETCS} level two duties is implemented.
This demonstration covers the compliance with a \abr{MA} by supervising the train's position, speed and braking curve, as well as the evaluation of \abr{RBC} messages and balise telegrams.
The on-board unit estimates the train's position based on on-board sensors and administers a confidence interval besides the estimated position.
A confidence interval specifies a range in which the train is located and is required because of possible inaccuracy in the position sensors.
In order to adjust the train's position and reset the confidence interval, balise telegrams are used.
Therefore, a list of balises and their positions that the train will encounter during its journey is communicated at the journey's start.
This is called the linking phase and the list of balises and balise position is referred to as linked balise groups.
While the \abr{MA} and linked balise groups are transmitted to the system at the start of the journey, balise telegrams must be reliably evaluated at any time.
In the course of this work, a flat track and a constant speed restriction is assumed.
Therefore, any track gradient can be neglected and the train's deceleration only depends on its speed.
From this, the following tasks can be derived for the showcases system to comply with the \abr{ETCS} subset:

\begin{itemize}
\item Receive and evaluate \abr{MA} with a start and an end position
\item Receive and evaluate linking information pre-journey
\item Continuously calculate the train's position and confidence interval
\item Continuously calculate the train's braking curve based on its characteristics and the current \abr{MA}
\item Receive and evaluate balise telegrams during journey to adjust the train's position and confidence interval
\item Stop the train when end of \abr{MA} is reached, a linked balise group is not encountered at its position, or an unlinked balise group is encountered
\end{itemize}

The system's practicability will be demonstrated through a simulated scenario, which is described in~\autoref{subsec:ScenarioSimulation}.

\section{Consensus Algorithm}
\label{sec:ImpConsensusAlgorithm}

In this section, a consensus algorithm that follows the concepts of \texttt{Raft} and applies the \abr{DDS} publish/subscribe communication pattern, is presented.

In \texttt{Raft}, it is the leader's responsibility to send periodic heartbeat messages to notify followers about its existence.
The heartbeat messages, as well as all other messages in \texttt{Raft}, are expressed as \abrpl{RPC}.
At least two \abrpl{RPC} need to be supported, namely \texttt{AppendEntries} and \texttt{RequestVote}.
\texttt{AppendEntries} is used for sending heartbeat messages and other commands, called logs in \texttt{Raft}, while \texttt{RequestVote} is invoked by candidates to gather votes.

\begin{figure}[!hb]
	\centering
	\includegraphics[width=0.9\linewidth]{images/ThreeEUConsensusDDS}
	\caption{A raft consensus algorithm requires at least two types of \glsentryfull{RPC} to be passed in the system, namely \texttt{RequestVote} and \texttt{AppendEntries}. These two can be implemented using \glsentryfull{DDS} event topics.}
	\label{fig:ThreeRepConsensusDDS}
\end{figure}

\autoref{fig:ThreeRepConsensusDDS} shows how the concept of \texttt{Raft}, with its roles and \abrpl{RPC}, can be implemented within a pipelined approach using \abr{DDS}.
In this example, the replicas of type (A) publish their intermediate results in a \abr{DDS} event topic, which can be read by the individual voters.
Replica (4) with its corresponding voter took over the leader role and thereby controls the system's output by performing a majority voting over the intermediate results.
In addition, the leader publishes periodic heartbeat messages to the \texttt{AppendEntries} topic, which are received by the followers (5) and (6).
When a heartbeat message stayed out for a period, the system assumes that the leader has crashed and the followers initiate a new leader election using the \texttt{RequestVote} topic.
When a new leader is elected, it gains complete control over the output, subscribes to the \texttt{Replica A Result} topic, and starts sending heartbeat messages.
In the event of the previous leader coming back online again, it subscribes to \texttt{AppendEntries}, receives heartbeat messages and thereby takes over a follower role.
Thereby, crash faults (\textbf{F1}) for voters can be caught by the heartbeat mechanism.

The implemented pipelined approach, as constituted in~\autoref{fig:ThreeRepConsensusDDS}, is deployed on three execution units for cost and space reasons.
Further, the input frequency for the applied use-case is small compared to the system's throughput whereby a \ChallengeThrough is no problem.
Nevertheless, the approach can be spread out to more replicas by distributing the logic modules accordingly.
This is because the connecting \abr{DDS} topics allow free arrangement of the solution's logic modules.
\\

In order for the application to notice available data samples, \texttt{WaitSets} are used.
\texttt{WaitSets} are preferred over \texttt{Listeners} because they are state-based rather than event-based, which means they trigger as long as a certain state is present rather than when an event is triggered.
Thereby, it can be ensured that no available data is missed.
Further, by using \texttt{WaitSets}, the entire control about the application is managed by the application itself and no application code is executed in middleware threads.
\\

The system's state is stored as a global state in \abr{DDS} state topics and thereby managed by the middleware.
This ensures that all replicas can operate on the most recent system state and no replica becomes outdated.
Only the system's leader is permitted to alter the system's state, all other replicas have read-only access.
\\

The consensus algorithm guarantees that there is at most one leader present in the system at a time - i.e. no split brain - and that a new leader is elected when there is none present.
Time in \texttt{Raft} is subdivided into \textit{terms}, that each start with a leader election.
Each replica manages its own term as a private replica state.
Besides the current term, the replicas current \texttt{Raft} role and voting information is organized in the replica's private state.
The private state is kept up to date by exchanging messages with other replicas.

\subsection{Leader Election}
\begin{figure}[!hb]
	\centering
	\includegraphics[width=0.75\linewidth]{images/sequence/LeaderElection}
	\caption{When a replica notices that the leader crashed, it becomes a candidate, votes for itself and sends vote requests to all remaining replicas. When enough votes were granted, it becomes the new leader and starts sending heartbeats. Otherwise, it retries to become leader in the next term.}
	\label{fig:SeqLeaderElection}
\end{figure}

The presence of a leader in the system is detected through heartbeat messages.
When a replica does not receive a message from a leader for a specific period of time, called the \textit{election timeout}, it tries to become leader by initiating the voting process.
A high level overview about the voting process, after the previous leader crashed, is depicted in~\autoref{fig:SeqLeaderElection}.
Replica 1 notices the crashed leader first and tries to become the new leader by switching into candidate mode and sending a vote request to each remaining replica.
In the exemplary case in~\autoref{fig:SeqLeaderElection}, replica 2 accepts the vote request and replica 1 becomes the new leader.
It immediately starts sending heartbeat messages afterwards.
\\
In the following, the allocation, as well as the collection of votes, will be discussed in more detail.
\\\\
\begin{algorithm}[H]
\caption{Algorithm for vote allocation. Whether a vote gets granted or rejected depends on whether the replica that receives the vote request has already voted for another replica in the current voting's term.}\label{algo:VoteAllocation}
\SetKwData{Term}{\textit{term}}
\SetKwData{Sender}{\textit{senderID}}
\SetKwInOut{Input}{input}
\SetKwInOut{Output}{output}

\Input{A vote request with a \Term and a \Sender}
\Output{true if the vote for \Sender in \Term was granted, false otherwise}
\BlankLine
\If{\Term $<$ current term}{become follower in new \Term\;}
\If{\Term $==$ current term \textbf{AND} not voted in current term}{grant vote for \Sender\;}
\end{algorithm}

The allocation of votes is shown in~\autoref{algo:VoteAllocation}.
Each replica can only vote for a single other replica in each term.
Because the system's state is managed globally by \abr{DDS}, every replica is equally up to data so that vote allocation is implemented in a first-come-first-serve manner.
As soon as a vote request is received that has a higher term number than the replica's term, is automatically transitions into the follower state, because the system will have a new leader in the new term.
\\\\
\begin{algorithm}[H]\caption{Algorithm for vote collection. Only votes that were answered in the same term that the vote request was issued are considered. When enough votes are collected, the replica becomes leader. If a votes was answered in a more recent term, the vote collection gets aborted and the replica becomes a follower.}\label{algo:VoteCollection}
\SetKwData{VoteTerm}{\textit{voteTerm}}
\SetKwData{VoteGranted}{\textit{voteGranted}}
\SetKwInOut{Input}{input}
\SetKwInOut{Output}{output}

\Input{A vote reply with a \VoteTerm and a \VoteGranted flag}
\Output{Transition to either leader or follower state}glsentryfull
\BlankLine

\If{the replica is no candidate anymore}{return\;}
\If{\VoteTerm $>$ term when election started}{
become follower\;
return\;}
\If{\VoteTerm $==$ term when election started \textbf{AND} vote got granted}{
Increase number of granted votes in the election term\;
	\If{got enough votes}{
	become leader\;
	return\;}
}
\end{algorithm}

Whether a vote requests gets accepted or declined depends on the request's data and the replicas private state.
For each vote reply that is received by a replica, it is first checked whether the replica has left the candidate state in the meantime, which makes the entire vote reply irrelevant.
This can, for example, happen when another vote request with a higher term number has been granted while the replica waits for incoming vote replies itself.
Afterwards it is checked whether the term in which the vote request got processed by another replica is higher than the term in which the voting was started.
This fact indicates that the corresponding replica is outdated, so that it transitions into follower state and should not become a leader in the regarded voting process.
Finally, when the vote got granted and enough granted votes were received, the replica is elected as a new leader for the corresponding term.
The number of necessary voted depends on the total number of replicas in the system.

\begin{figure}[!hb]
	\centering
	\includegraphics[width=0.75\linewidth]{images/RaftServerStates}
	\caption{Taken from~\cite{RaftConsensusPaper}. Replicas in \texttt{Raft} can be in one of three states. When a follower receives no message from a leader, it starts an election and tries to become the new leader. Leaders operate until they fail or until they discover that their term is outdated.}
	\label{fig:RaftServerStates}
\end{figure}

During the entire time, each replica is in one of three states, namely \texttt{Leader}, \texttt{Candidate} or \texttt{Follower}.
The leader election process can be summarized as transitions and corresponding conditions between these three states, as depicted in~\autoref{fig:RaftServerStates}.
Five requirements can be derived from the figure that must be met by an algorithm that implements \texttt{Raft}'s leader election process:

\begin{enumerate}
\item \textbf{Start Election:} A follower becomes candidate, increments its term, votes for itself and sends a message to all other replicas stating that it wants to become leader.
\item \textbf{End of Election:} A candidate remains in candidate state until it either wins the election, receives information about another leader in the system, or times out.
\item \textbf{Won Election:} A candidate wins the election if it receives votes from a majority of replicas in the same term. Each replica votes for at most one replica in a given term.
\item \textbf{Lost Election:} When a candidate receives a message from a leader in the system and the leader's term is at least as high as the candidate's term.
\item \textbf{No Result:} A certain period of time goes by without the candidate winning or loosing the election.
\end{enumerate}
In the following, it is described how these requirements are met by using \abr{DDS} features.
Thereby, the functional correctness of the transcribed leader election algorithm is proven.

\lstset{language=C}
\begin{lstlisting}[caption={\abr{IDL} definition for the \texttt{AppendEntries} topic. The \texttt{term} variable represents the latest term that the replica has seen, while the \texttt{senderID} encodes which replica sent this message. With \texttt{entries}, a payload can be send via the topic. This is used by a leader for instructing its followers to process certain data. The \texttt{entries} field is left empty for heartbeat messages.}, label=code:appendEntries]
struct AppendEntries {
    long term;
    long senderID;
    sequence<Entry> entries;
};
#pragma keylist AppendEntries term
\end{lstlisting}

\begin{lstlisting}[caption={\abr{IDL} definition for the \texttt{RequestVote} topic. The \texttt{term} variable represents the candidate's term, while \texttt{candidateID} encodes the candidate that requested the vote.}, label=code:requestVote]
struct RequestVote {
    long term;
    long candidateID;
};
#pragma keylist RequestVote
\end{lstlisting}

\begin{lstlisting}[caption={\abr{IDL} definition for the \texttt{RequestVoteReply} topic. The \texttt{term} encodes the sender's term for the candidate to update itself. \texttt{voteGranted} shows whether the replica granted the vote request for the given term. The \texttt{candidateID} and \texttt{senderID} are used to identify which replica granted the vote for which replica respectively.}, label=code:requestVoteReply]
struct RequestVoteReply {
    long senderID;
    long term;
    long candidateID;
    long voteGranted;
};
#pragma keylist RequestVoteReply
\end{lstlisting}

For realizing the leader election algorithm with \abr{DCPS} features, three \texttt{Topics} are required.
The \texttt{AppendEntries} and \texttt{RequestVote} topics are representatives for the corresponding \texttt{Raft} \abrpl{RPC}.
The \abr{IDL} representation for \texttt{AppendEntries} is shown in~\autoref{code:appendEntries}.
In accordance to that, \texttt{RequestVote} is represented by a topic whose \abr{IDL} is shown in~\autoref{code:requestVote}.
Because data objects that are managed by \abr{DDS} require a predetermined structure, a dedicated topic for replying to vote requests is required.
Therefore, \texttt{RequestVoteReply} is utilized as described in~\autoref{code:requestVoteReply}.

For implementing an \textit{election timeout}, a \texttt{WaitSet} with a corresponding \texttt{ReadCondition}, that triggers when new data has been published to \texttt{AppendEntries}, is applied.
Although a heartbeat message is a periodic message and a deadline \abr{QOS} could be used to register an absent leader, a timeout attached to a \texttt{WaitSet} is preferred.
This is because a \texttt{WaitSet} with a timeout is required for other messages as well and the deadline \abr{QOS} would require a \texttt{StatusCondition} to be noticed whereby the utilized \abr{DDS} subset would grow.
In order to initiate the voting process, the replica publishes a new vote request to \texttt{RequestVote}.
Because the middleware ensures that data is transmitted to all replicas that subscribed to a topic, the \textbf{Start Election} requirement can easily be solved with \abr{DDS}.

Further, \textbf{No Result} can be resolved by using a \texttt{WaitSet}, called \textit{leaderElection\_WaitSet}, with a corresponding timeout (\textit{leader ready timeout}).
When the timeout expires, the replica starts to wait for heartbeat messages again.
In addition to the timeout, a \texttt{ReadCondition} is attached to the \textit{leaderElection\_WaitSet} that triggers when messages from a leader have been published to the \texttt{AppendEntries} topic.
Thereby, \textbf{Lost Election} is ensured.
By implication, this means that the candidate either won the election, or the election was not successful for the turn, when the timeout attached to \textit{leaderElection\_WaitSet} expires.

However, because each replica waits for incoming heartbeats after it won the election, a leader could simultaneously read its own heartbeats.
This is because each replica has simultaneously registered a \texttt{DataWriter} and \texttt{DataReader} to each topic.
Therefore, after reading each data sample, the application code must manually verify that the message was not published by the same replica that is reading the message by comparing its identification number with the received \textit{senderID}.
Because it depends on the replica's role whether it should publish or subscribe to a topic and each replica could become the leader or a follower at any point in time, all \abr{DDS} entities and subscriptions are set up during application initialization.
Thereby, time is saved during role reversal.
On the other hand, this also has the effect that each replica has \texttt{DataReaders} that receive messages published by \texttt{DataWriters} from the same replica.
While \texttt{QueryConditions} could be used to solve this by filtering messages, \texttt{ReadConditions} have been used because it ensures that each data sample is read and \texttt{DataReaders} with resource limit \abr{QOS} settings are not overfilling.
This partly solves \textbf{Won Election}.

In order to fully solve \textbf{Won Election}, each replica needs to collect votes and reply to vote requests, parallel to detecting other leaders and voting timeouts.
The voting part is therefore treated by another \abr{OS} thread, that again makes use of a \texttt{WaitSet}, where two \texttt{QueryConditions} are attached to, one for the \texttt{RequestVote} and one for the \texttt{RequestVoteReply} topic.
Thereby, the replicas can come to an agreement about a vote request while the candidate waits for the voting to end.

\subsubsection{Race Conditions}
\begin{figure}[!hb]
	\centering
	\includegraphics[width=0.75\linewidth]{images/LeaderElectionHeartbeatThread}
	\caption{One \glsentryfull{OS} thread of the leader election process is the "Heartbeat Thread". In this thread, followers receive \texttt{AppendEntries} messages from leaders and detect missing leaders through expiring timeouts. The thread also initiates the voting process by sending an initial vote request. A mutex ensures that the algorithm can only be preempted before or after a part that is depicted in green. Consecutive red parts are executed atomically.}
	\label{fig:LeaderElectionHeartbeatThread}
\end{figure}

\begin{figure}[!hb]
	\centering
	\includegraphics[width=0.75\linewidth]{images/LeaderElectionVoteThread}
	\caption{One \glsentryfull{OS} thread of the leader election process is the "Vote Thread". In this thread, replicas respond to vote requests and handle replies they got for their own vote requests. A replica's action to the requests depends on their private state. A mutex ensures that the algorithm can only be preempted before or after a part that is depicted in green. Consecutive red parts are executed atomically.}
	\label{fig:LeaderElectionVoteThread}
\end{figure}

The leader election algorithm is structured into three parts which are executed on two \abr{OS} threads.
One part is the processing of heartbeat messages in a "Heartbeat Thread, as depicted in~\autoref{fig:LeaderElectionHeartbeatThread}.
In the "Heartbeat Thread", each follower listens for heartbeat messages and start a leader election process by sending vote requests when the corresponding timeout expires.
The other two parts, namely answering vote requests, and processing replies to vote requests, are handled in another thread and are depicted in~\autoref{fig:LeaderElectionVoteThread}.
Since the program is structured as a multi-threaded concurrent system that is required to make decision based on the replica state and no assumptions about the thread's relative speed can be made, race conditions can occur~\cite{Dijkstra1965}.
Replica internal race conditions are prevented by utilizing a mutex in order to ensure that its private state, including the replica's \texttt{Raft} role, the current term, and voting information, does not change while answering a message.
Race conditions among replicas are prevented though a history \abr{QOS}-policy and a continuously increasing term number.
The history \abr{QOS}-policy ensures that new requests are buffered in a \texttt{DataReader} while another is processed.
This ensures that no request is neglected.
Further, a continuously increasing term number enables the system to detect outdated requests or recognize when the replica itself is outdated.


\subsubsection{Time Considerations}
\label{subsub:timeConsiderations}
Because each \texttt{WaitSet} in \abr{DDS} can be attached with a timeout, the maximal time that the system is without a leader can be determined.
This is further measured and analyzed in~\autoref{chptr:evaluation}.
There are two timeouts involved in the leader election process, namely the \textit{election timeout}, whereby a missing leader is elected, and the \textit{leader ready timeout}.
When the \textit{leader ready timeout} expires, it was impossible for the system to elect a new leader for the given term.
This can, for example, happen due split votes when multiple followers try to become the new leader at the same time, or because there are not enough active replicas in the system.
The proposal that \texttt{Raft} makes to reduce the risk of split votes is to use randomized \textit{election timeouts}.
However, this does not prevent split votes from happen at all.
Therefore, in this implementation, the \textit{election timeout} is made directly dependent on the replicas unique identification number, so that they never try to become leader at the same time.
Although replicas with certain identification numbers are thereby preferred in the leader election process, this approach is not a disadvantage because, due to \abr{DDS}'s global data space, no replica is ever more suitable for becoming the new leader than some other.
Further, a replica with an outdated term number is still excluded from becoming a leader because their vote requests would get declined.

On the other hand, network delays or too few active replicas are safety risks and should be dealt with accordingly.
For doing so, a maximal time that the system should be without a leader can be specified.
Altogether, when enough replica are present in the system and communication is possible without any delays, the system will not be without a leader for the maximum time of $\textit{election timeout} + \textit{leader ready timeout}$.

\subsection{Input Processing}
\label{subsec:ImpInputProcessing}
\begin{figure}[!hb]
	\centering
	\includegraphics[width=0.75\linewidth]{images/sequence/CollectResults}
	\caption{Input messages from track-side components are recorded by the leader and replicated to its followers. Afterwards, the leader and each follower process the input and generate a decision based on the data and the global system state. The decisions are collected by the leader and a voting is performed on the collected decisions. Meanwhile, the leader ensures that time constraints are met using timeouts.}
	\label{fig:SeqCollectResults}
\end{figure}

The purpose of the established redundant application is to process input messages from \abr{RBC} and track-side components and generate a single reaction on the input for the entire system based on majority voting.
A coarse overview about how input messages are distributed among the replicas and processed by the system is presented in~\autoref{fig:SeqCollectResults}.
It is the leaders responsibility to record new inputs, replicate the inputs to each follower, maintain the system's global state, and generate a final output to each input.
All communication among the replicas takes place via \abr{DDS}.

New messages on the \texttt{Input} topic are recognized with a \texttt{ReadCondition} that is attached to a \texttt{WaitSet}.
As soon as the leader receives an input messages, it publishes a new command via the \texttt{AppendEntries} topic and attaches the input message to the \texttt{entries} field.
Thereupon, when the followers receive this message, they start to generate a decision for the input message based on the global system state which is managed by \abr{DDS} and publish their decisions to the \texttt{AppendEntriesReply} topic.
Concurrently, the leader also generates a decision and starts collecting the follower's decisions using a \texttt{WaitSet} - called \texttt{AppendEntriesReply\_WaitSet} - and a \texttt{QueryCondition}.
As soon as all followers answered, the leader performs a voting on the collected decisions.

Further, a timeout is attached to the \texttt{AppendEntriesReply\_WaitSet} to ensure that the voting process is not deferred indefinately.
A voting can be conducted when half of the followers or more answered during this time-span.
As an example, when one follower answered in a \abr{TMR} setup, two decisions are available for the leader, namely his own and one follower decision.
Both need to correspond in order to get a valid result.
When these two differ, the more restrictive one needs to be choosen.
When less than half of the followers answered during the time-span, the system needs to transition into a safe state.

After the leader successfully performed a voting, it commits the final decision.
This is mandatory because the decision can have an effect on the system's global state, for example when a balise has been crossed and the train's confidence interval is resetted.
Furthermore, the input message is marked as processed by disposing the corresponding \abr{DDS} instance.
This ensures that if the leader is deselected or crashed while processing the input, the input will be read and processed by the next leader.

\paragraph{Decision Making}

\todo{Detlich machen, dass man an Safety in verschiedenen Szenarien gedacht hat}

\begin{figure}[!hb]
	\centering
	\includegraphics[width=0.75\linewidth]{images/DecisionMaking}
	\caption{To decide whether the train should brake or can continue its journey, this algorithm is executed. When the train is not driving, it should brake. IF a balise telegram is evaluated, it is checked whether the balise is linked and whether the calculated train position corresponds with the balise's expected position. Finally, the train's braking curve is calculated and it is checked, whether the train would reach the current \glsentryfull{MA}'s end position when it would brake in this moment. Only if everything goes as planned may the train continue its journey.}
	\label{fig:DecisionMaking}
\end{figure}

The exemplary \abr{ETCS} subset, as introduced in~\autoref{sec:ImpScenarioDescription}, requires processing of \abr{MA}-messages, linking information, and balise telegrams.
Messages regarding \abrpl{MA} and linking information are directly processed by the leader without involving the followers.
That is because these messages do not need a system-wide decision, the information they contain just needs to be added to the global state.
The balise telegrams, however, require evaluations that can have safety-related effects and are therefore processed by all replicas in the system.
To make these decisions, the algorithm that is depicted in~\autoref{fig:DecisionMaking} is executed.
The algorithm includes various conditions that are processed one after the other.
If the train is not driving at all, it should brake.
When the decision should be made based on a balise telegram, it is checked whether the balise is linked and the train's calculated position corresponds with the balise's expected position.
If both are the case, the train's braking curve is calculated based on its speed and deceleration properties.
The braking distance is added to the train's most recent position and compared to the current \abr{MA}'s end position.
If the train would reach this position, the braking process is initiated.

In case only the periodic monitoring of the braking curve is due and therefore no balise telegram is available for evaluation, the calculation is continued directly with the braking curve evaluation.

\subsubsection{Time Considerations}
Two time-related issues are critical to input processing.
First, a result is expected for each input after a certain period of time.
Second, the leader expects the followers' decision after a certain time.

\todo{Check how the first time requirement can be solved}
The second time requirement can be solved by attaching a timeout to the \texttt{AppendEntriesReply\_WaitSet}.

A third timeout is attached to the \texttt{Input} topic's \texttt{WaitSet} and does not solve time requirements, but is utilized to periodically trigger a system wide braking curve check.
This ensures that if no input is ready, the braking curve is monitored periodically by sending a certain message via the \texttt{AppendEntries} topic when the timeout expires.

\subsection{State Manager}

\begin{lstlisting}[caption={\abr{IDL} definition for the \texttt{LinkedBalises} topic. Each linked balise has an unique identifier and a position that is communicated to the system by the \abr{RBC}.}, label=code:linkedBalises]
struct LinkedBalises {
    short ID;
    long position;
};
#pragma keylist LinkedBalises ID
\end{lstlisting}

\begin{lstlisting}[caption={\abr{IDL} definition for the \texttt{MovementAuthority} topic. The \texttt{start\_position} encodes where the \abr{MA} starts and the \texttt{end\_position} encodes until where it is valid.}, label=code:movementAuthority]
struct MovementAuthority {
    long start_position;
    long end_position;
};
#pragma keylist MovementAuthority
\end{lstlisting}

\begin{lstlisting}[caption={\abr{IDL} definition for the \texttt{TrainState} topic. The train's state consists of a current position and a current speed. Due to inaccuracies of the position sensors, a train's position cannot be determined exactly. Therefore, a confidence interval is maintained that defines an area where the train certainly is. This area is bounded by \texttt{max\_position} and \texttt{min\_position}. With \texttt{is\_driving} it is encoded whether the virtual train drives or stands still. The \texttt{lastUpdateTime} variable is used to simulate the train's position based on its speed.}, label=code:trainState]
struct TrainState {
    double position;
    double max_position;
    double min_position;
    double speed;
    boolean is_driving;
    unsigned long long lastUpdateTime;
};
#pragma keylist TrainState
\end{lstlisting}

It is mandatory for the replicas in the distributed system to produce deterministic results and to have access to the most recent global system state.
The global system state consists, inter alia, of the current \abr{MA} and a set of linked balises.
Although the train's position and speed are typically administered by other dedicated components in actual operation, they are simulated for the demonstrated system and therefore part of its global system state.
\\

\abr{DDS} is used to manage the system's global state and ensure that any change is reliably trasmitted to all replicas.
Therefore, three \texttt{Topics} are used, namely \texttt{LinkedBalises}, \texttt{MovementAuthority}, and \texttt{TrainState}.

The \abr{IDL} representation for the \texttt{LinkedBalises} topic is shown in~\autoref{code:linkedBalises}.
Multiple balise instances can be stored within the topic that are distinguished by a unique identification number.
Since a set of linked balises is sent before journey and is not expected to change afterwards, an enduring \abr{DDS} state topic is used (see~\autoref{tab:stateQOS}).

The currently valid \abr{MA} is stored in the \texttt{MovementAuthority} topic, whose \abr{IDL} is shown in~\autoref{code:movementAuthority}.
As with the \texttt{LinkedBalises} topic, a movement authority is assigned once and valid for an entire journey so that it is stored in an enduring state topic.
Only one \abr{MA} is expected to be stored in the topic at any time.

The train's position and speed are stored in the \texttt{TrainState} topic, whose \abr{IDL} is shown in~\autoref{code:trainState}.
Three variables are used for the position.
One encodes the estimated position (\textit{position}) and two delimit an interval, the so called confidence interval, that certainly contains the train's true position.
A confidence interval is required because position measuring instruments can have inaccuracies.
The topic further contains a \texttt{is\_driving} variable that indicates whether the train moves or stands still and a \textit{lastUpdateTime} variable that determines when the position was updated for the last time.
This \texttt{lastUpdateTime} variable is only used for simulation reasons.
Because the state data is updated periodically and frequently, it should be managed as a volatile \abr{DDS} state topic.
\\

The topic's data is read and written like any other \abr{DDS} topic.
However, when a train starts a new journey and the new track section's linked balises have identification numbers that correspond with identification numbers of linked balises from previous track sections, special treatment is required.
In order to circumvent this situation beforehand, any data on the \texttt{LinkedBalises} topic is disposed before new data is written.

\section{Scenario Simulation}
\label{subsec:ScenarioSimulation}

\begin{figure}[!hb]
	\centering
	\includegraphics[width=0.75\linewidth]{images/SimulatorWorkflow}
	\caption{The scenario simulation's workflow. Each simulation begins by creating a scenario from a file and sending \glsentryfull{MA} and linking information. Thereupon, the virtual train is set in motion and balise telegrams are send on its way. When the virtual train stops early or when the \abr{MA}'s end is reached, the simulation stops.}
	\label{fig:SimulatorWorkflow}
\end{figure}

In order to demonstrate and evaluate the system's applicability, the \abr{ETCS} subset from~\autoref{sec:ImpScenarioDescription} is simulated in form of scenarios.
The \abr{ETCS} subset consists of a \abr{MA} and a list of balises that function as positional landmarks.
Thus, a simulated scenario is bounded by a \abr{MA}'s start- and end-position and consits of one or multiple balises.
A simulated scenario's workflow is depicted in~\autoref{fig:SimulatorWorkflow}.

\begin{lstlisting}[caption={A JSON scenario representation. Each scenario consists of a \glsentryfull{MA} with a start and end point, as well as of a set of balises. Balises can be either linked or not and consist of an identification number, a linking position and an actual position. The linking position is transmitted to the on-board unit and is distinguished from its actual position to simulate a misplaced balise.}, label=code:scenarioJSON]
{
  "MA": {
    "start": 0,
    "end": 10
  },
  "balises": {

    "balise": {
      "id": 0,
      "link_pos": 1,
      "pos": 1,
      "linked": true
    },
    "balise": {
      "id": 1,
      "link_pos": 5,
      "pos": 5,
      "linked": true
    },
    "balise": {
      "id": 2,
      "pos": 7,
      "linked": false
    }

  }
}

\end{lstlisting}

At first, a scenario is created from a \textit{json} file.
An exemplary scenario in \textit{json} format is shown in~\autoref{code:scenarioJSON}.
It contains a \abr{MA} that starts at the current position and is valid for 1000 meters.
Further, three balises are simulated on the way, one at 100 meters, one at 500 meters, and one at 700 meters after the journey's start.
The first two balises will be linked with the on-board unit while the third will not.
Any linked balise further has a linking position field which is communicated to the on-board unit.
The distinction between a balise's actual position and its linking position is used to simulate balise that are not where they should be.
\\

After the \abr{MA} and the two linked balises have been sent to the on-board unit, the simulator starts to simulate a trip by updating the virtual train's position.
The position is continuously compared to the balise's positions and the \abr{MA}'s end.
As soon as the simulated train passes a balise's position, the corresponding balise's data is published to the \texttt{Input} topic.
In case that the balise lead to a braking action, for example because the balise was not linked or its position and linking position do not correspond, the simulation stops.
Finally, when all balises are evaluated successfully, the simulation ends when the \abr{MA}'s end is reached.
\\

\paragraph{Simulator Classification}
In literature, simulators are distinguished on the basis of their mode of operation and categorized into \abr{DE}-, and \abr{CT}-based simulators~\cite{CoSimulationStateOfTheArt}.
In a \abr{DE}-based simulation, communication with the environment is characterized by events that are triggered at certain times.
For \abr{CT}-based simulations, the simulated state is expected to evolve continuously over time.
Based on these properties, it can be argued that the simulator that is utilized in this work is a combination of a \abr{DE}- and a \abr{CT}-based simulation system.
On the one hand, the train's movement is simulated in a way that is typical to \abr{CT}-bases simulations, because the train's position continuously evolves over time based on its speed.
On the other hand, the track-side balise telegram simulation borrows features from \abr{DE}-based simulators, because it communicates via events.
The events are ied to the simulated train's position and are thereby time-stamp based.

\section{Hot Standby}
\begin{lstlisting}[caption={\abr{IDL} definition for the \texttt{ActivateSpare} topic. This topic is used to activate or deactivate spare replicas. The \texttt{term} field encodes the term in which the activate or deactivate call has been made and \texttt{activate} gets interpreted as a boolean that encodes whether the spare should be activated or deactivated.}, label=code:activateSpare]
struct ActivateSpare {
    long term;
    long activate;
};
#pragma keylist ActivateSpare
\end{lstlisting}

\begin{figure}[!hb]
	\centering
	\includegraphics[width=0.75\linewidth]{images/sequence/ActivateSpare}
	\caption{When the leader, after reading an input message and instructing its followers to process the input, receives too few decisions after a certain time, it activates the spare replica. When the follower receives more than necessary, the spare component gets deactivated again.} as well as replies to vote requests
	\label{fig:SeqActivateSpare}
\end{figure}

A method for adding hot standby spare replicas to the system using \abr{DDS} topics has been depicted in~\autoref{fig:TMRWithSparesDDS}.
To put this approach into practice, a new topic was introduces, called \texttt{ActivateSpare}, whose \abr{IDL} is shown in~\autoref{code:activateSpare}.
Any activate and deactivate message is sent via the \texttt{ActivateSpare} topic.
The additional logic merges into the log replication logic and is depicted in~\autoref{fig:SeqActivateSpare}.
When collecting all decisions from the follower replicas, the leader activates or deactivates the spare component when receiving too few or too much decisions, respectively.
Any spare replica is initialized in spare mode and thereby excluded from making decisions, from becoming a candidate or leader, and from participating in the leader election process.
After being activated by the leader, the spare component transitions into follower state and can therefore make decisions when asked by the leader and vote for new leaders during the leader election process.
%However, since they are required to be deactivated when too many replicas are active, any replica that has been a spare component is excluded from becoming the system's leader.

While being in \texttt{Spare} state, a replica utilizes a \texttt{WaitSet} to wait for any activate message.
When an activated spare component receives a message from the leader that should be processed, it checks whether the leader has deactivated the spare beforehand.
When this is the case, it transitions into \texttt{Spare} mode again and discards any other message.
Because of timing issues such as network delays, it can happen that an activated spare receives a \textit{deactivate} message after it processed the leader's order.
Therefore, it can happen that the leader gets more replies than there are active components in the system.
It is the leader's responsibility to handle these additional decisions appropriately.
\\

Only a single spare component is currently supported in the exemplary implementation.
However, multiple spare components can easily be added by introducing a \texttt{spareID} field in the topic's data that encodes the identification number for the spare to activate.


% TODO Further specify
\iffalse

However, without any further treatment, only crash faults can be tolerated for voters in this setup.
Methods for detecting and handling time and omission faults exist but require additional hardware.
However, since the leader has sole control about the system's output, it is impossible to tolerate computation faults in this way.
Thus, it needs to be assured that neither a voter, nor a replica of type (B), have any omission, timing, or computation fault.



Using replicated voters with a consensus algorithm, \ChallengeVoter can be solved, because the voter is no single point of failure anymore.
\fi


    \chapter{Evaluation}
\label{cpt:evaluation}
In this chapter, the developed system is evaluated with respect to various criteria.
At first, an overview about the utilized \abr{DDS} subset is described and the necessity of each feature is justified.

This is followed by an system evaluation where both the system's hardware structure as well as the implemented software are evaluated.
Within the implementation evaluation, all characteristics that correspond with the manner of how the software is implemented, are considered.
First, general system properties are analyzed, such as the time needed to send and receive messages or the maximum time the system can be without a leader.
This allows to answer fundamental questions about the system's reliability because the average time to synchronize the components can be derived.
Afterwards, metrics related to the execution of specific scenarios are examined.
These include the utilization of system resources on the one hand and the number of messages sent and received on the other.
From these metrics it can be seen at which points in time an increase in system performance can be expected and that enough resources are always available in the applied system.

\section{System Evaluation}
At first in this section, general system metrics are analyzed.
Afterwards, metrics related to concrete, simulated scenarios are considered.
For this purpose, three scenarios are processed one after the other by the redundant system and certain data are collected during this process.

\subsection{General System Characteristics}
General system properties include those that are independent of environmental influences.
This includes the time it takes the system to elect a new leader, from which the maximum time that the system without a leader can be derived.
The election time is limited by the replica's computational resources and by the speed of data transmission.
The latter is investigated in the next step by publishing and reading messages of different sizes through \abr{DDS} topics.
Finally, the resource utilization of the system in idle mode is examined.
Idle means that the train is at a standstill and the system is ready to start the journey.

\paragraph{Leader Election Time}
\todo{Redo measurements}
\begin{figure}[!hb]
	\centering
	\includegraphics[width=0.75\linewidth]{images/plots/timeWithoutLeader}
	\caption{Time that the system is without a leader in each term for 20 terms. The average of 5472µs is marked by the red line.}
	\label{fig:PlotTimeWithoutLeader}
\end{figure}

When the system's leader crashes, the current \texttt{Raft} term ends and a new leader gets elected.
The leader election process consists of requesting and collecting votes from all replicas in the system.
In order to investigate how long the leader election process takes, the time for 20 elections was measured.
This has been done by starting a timer when a missing leader got noticed until a heartbeat message from a new leader was received.
The results can be seen in~\autoref{fig:PlotTimeWithoutLeader} and show that the process takes around 5500µs on average and at worst 7000µs.
However, since the system utilizes a timeout to detect an absent leader, the total time that the system is without a leader arises from the sum of the applied election timeout and the election duration from~\autoref{fig:PlotTimeWithoutLeader}.

\paragraph{Data Transmission Time}
\begin{figure}[!hb]
	\centering
	\includegraphics[width=0.75\linewidth]{images/plots/sendingTimes}
	\caption{Time that is required to transmit messages with different payload sizes via a \abr{DDS} topic. The times were determined by calculating the mean sending time out of 200 messages.}
	\label{fig:PlotSendingTimes}
\end{figure}


A end-to-end probing approach has been chosen over a \abr{IP} \abr{TTL} technique, which has been proposed by~\cite{SinhaMeasureNetworkLatency}, because the applied network has the same diameter for each node.
When a component publishes data to a \abr{DDS} topic, the middleware ensures that the data is transmitted to all registered subscribers.
In the system that is deployed for this thesis, data is transmitted via an \textit{Ethernet} connection.
The replicas are interconnected via a network switch that allows up to 2000 Mbit per seconds for each port.
In order to verify the applied topology the time that is required to send a message with a certain payload size between two replicas is measured.
For bypassing clock skews in the system, a request/response approach is used, where one replica publishes a message with a certain payload size and waits until a receiving replica confirms the message's receipt.
The receiving replica constantly checks for new messages and, as soon as it receives some, publishes a confirmation messages on another topic.
Any time that is required to send the confirmation message is negligible because it only consists of a 8-bit identification number, that assigns it to a payload carrying message.
The results, which are shown in~\cite{fig:PlotSendingTimes}, were determined by measuring the time that elapsed between sending a message and receiving the corresponding confirmation.
This process was repeated 200 times for each payload size and a mean was calculated.
What emerged is, that the transmission time is directly dependent on the published message's size.
\\

\begin{table}[h!]
	\begin{center}
		\caption{All topics that are utilized in the system have a certain data schema. From the resulting message size, the transmission time in the system can be calculated. The size and transmission time of the \texttt{AppendEntries} and \texttt{Input} depends on the length of the data sequence (\textbf{l}).}
		\label{tab:topicSendingTimes}
		\begin{tabularx}{\textwidth}{|X|X|X|}
			\hline
			\textbf{Topic} & \textbf{Message Size in Bytes} & \textbf{Expected Transmission Time in µs} \\
			\hline \hline
			AppendEntries & $12 + 4 * l$ & $0.402768 * l + 3219.11$ \\
			\hline
			AppendEntriesReply & 18 & 3219.7 \\
			\hline
			RequestVote & 8 & 3218.7 \\
			\hline
			RequestVoteReply & 13 & 3219.21 \\
			\hline
			Input & $4 + 4 * l$ & $0.402768 * l + 3218.3$ \\
			\hline
			ActivateSpare & 5 & 3218.40 \\
			\hline
			LinkedBalises & 6 & 3218.5 \\
			\hline
			TrainState & 41 & 3222 \\
			\hline
			MovementAuthority & 8 & 3218.7 \\
			\hline
		\end{tabularx}
	\end{center}
\end{table}

By the data size of the messages for the topics used in the system, the expected transmission times can be calculated.
These are listed in the~\autoref{tab:topicSendingTimes}.
The sizes are based on the \textit{Revolution Pi}'s system characteristics.
In ADLINK's \texttt{OpenSplice DDS}, the \textit{short} \abr{IDL} datatype resolves to a \textit{short int} C type, \textit{long} resolves to \textit{int} for the C programming language, \textit{unsigned long long} resolves to \textit{unsigned long long int}, and the \textit{boolean} datatype resolves to an \textit{unsigned char}.
A \textit{double} \abr{IDL} datatype resolves to an \textbf{double} C type.
The replica system's size characteristics for an \textit{short int} are two bytes, for an \textit{int} are four bytes, for an \textit{unsigned long long int} are eight bytes, for a \textit{double} are eight bytes, and for an \textit{unsigned char} is one byte.
The transmission time was calculated via the equation $transmissionTime(size) = 0.1100692 * size + 3217.9$ that has been derived by regression from~\autoref{fig:PlotSendingTimes}.
For the \texttt{AppendEntries} and \texttt{Input} topic, the actual data size and thereby the expected transmission time depend on the data sequence's length. 

\todo{Was bedeutet das für das konkrete System}

\paragraph{Idle Resource Utilization}
\begin{figure}[!hb]
	\centering
	\includegraphics[width=0.75\linewidth]{images/plots/CPUUsageIdleTime}
	\caption{This plot shows the \glsentryfull{CPU} utilization in \% for both leader and follower in idle mode where only heartbeat messages are exchanged.}
	\label{fig:PlotCPUUsageIdleTime}
\end{figure}

In this paragraph, the utilized \abr{CPU} resources for leaders and followers are examined in both idle mode and when the system receives input messages.
Measurements were made with \textit{pidstat}, a tool that monitors individual tasks that are managed by the linux kernel.
\\

In idle mode, the train is not driving and only heartbeat messages are exchanged between the leader and its followers.
The \abr{CPU} usage for both a leader and a follower in idle mode are depicted in~\autoref{fig:PlotCPUUsageIdleTime}.
It can be seen that the utilization rate is consistently higher for the leader than for a follower.
On average, the follower utilized 1.81\% in 90 seconds, while the leader utilized 2.42\% of the applied \abr{CPU}.
In system mode, 0.21\% are used for the leader and 0.25\% for a follower on average.
For user mode, the leader utilizes on average 1.0\%, while a follower uses 0.66\%.
This can be explained by the fact that the leader has more responsibilities in idle mode.
It not only periodically sends heartbeat messages, but also observes whether the train has a valid \abr{MA} and can start driving.


\subsection{Scenario Simulation}

\begin{table}[h!]
	\begin{center}
		\caption{A Python program is used for automatic integration testing. The scenarios that are traversed in the process differ in the relation of linked and unlinked balises and their expected outcome.}
		\label{tab:simulatedScenarios}
		\begin{tabularx}{\textwidth}{|X|X|X|}
			\hline
			\textbf{Name} & \textbf{Number of balises/linked balises} & \textbf{Expected behaviour}\\
			\hline \hline
			Reach End of \abr{MA} & Three/Three & All balise telegrams are evaluated correctly and the train stops before the \abr{MA} ends. \\
			\hline
			Unlinked Balise & Three/Two & The train stops when the unlinked balise is encountered. \\
			\hline
			Balise not where expected & Three/Three & The actual position of one balise does not correspond with its linked position. The train stops when this balise is encountered. \\
			\hline
		\end{tabularx}
	\end{center}
\end{table}

For integration testing and evaluation, three scenarios are simulated using the simulator that was described in~\autoref{subsec:ScenarioSimulation}.
All three scenarios include a \abr{MA} of equal size and three balises.
However, they differ in the number and position of their linked balises.
An overview about the simulated scenarios and the system's expected behaviour is given in~\autoref{tab:simulatedScenarios}.
\\

For evaluation purpose, a functionality that writes the system's reactions to external messages into an evaluation file was patched into the on-board unit software.
Said external factors include \abr{RBC} messages containing linking and \abr{MA} information, balise telegrams, as well as braking curve monitoring.
Every time the system reacts, the current position, the performed action, the encountered balise, and the reason for the action are listed.
A python script, that invokes the simulator and assesses the system's evaluation file, is used for automated integration testing.
Thereby, a data set is generated that contains a record for each data sample that was published to or written from a topic in the system.

In a second experiment, the system's hardware resource utilization is recorded.
This data set provides information on whether the used hardware components can withstand the demands of the implemented system.
\\

Because the trips are simulated, a reproducible test environment is created.
Nevertheless, the implemented system is asynchronous and thereby the component synchronization process is not deterministic.
Further, the replica's hardware resources might not meet the scenarios and system's requirements.
For identifying whether the applied hardware resources are sufficient, the system's resource utilization is assessed.
Finally, the messages that were exchanged between system components during the scenario simulation are analyzed and visualize in order to show the system's behaviour.
All following diagrams are based on measurements that were taken while the system processed simulated scenarios.
The data was processed by a python script using \textit{matplotlib}.

\paragraph{Resource Utilization in Operation}
\todo{Add memory consumption}

\begin{figure}[!hb]
	\centering
	\includegraphics[width=0.75\linewidth]{images/plots/CPUUsage}
	\caption{This plot shows the \glsentryfull{CPU} utilization in \% for both leader and follower while the system receives and evaluates input messages.}
	\label{fig:PlotCPUUsage}
\end{figure}

The \abr{CPU} utilization when the system receives and evaluates input messages is depicted in~\autoref{fig:PlotCPUUsage}.
For this measurement, three scenarios, namely \texttt{Reach End of \abr{MA}}, \texttt{Unlinked Balise}, and \texttt{Balise not where expected} where run and \abr{CPU} utilization has been measured with \texttt{pidstat}.
On average, the leader utilizes 7.4\% of the applied \abr{CPU} resources, while a follower utilizes 12.675\% on average.
In system mode, 1.65\% are used for the leader and 1.05\% for a follower on average.
For user mode, the leader utilizes on average 5.73\%, while a follower uses 11.63\%.
\todo{Explain why follower might have higher CPU utilization}

The first scenario, \texttt{Reach End of \abr{MA}}, starts at 0 seconds.
The leader's first peak after two seconds (S1) is due to the balise and linking information of the first scenario being evaluated.
At the same time, the followers compare the train's braking curve with the new \abr{MA}, for which the \texttt{MovementAuthority} topic is accessed and the first peak on the follower side can be explained.
After four seconds, the follower has another peak in \abr{CPU} utilization.
At this point, the first balise is encountered and the train's position is compared to the linked balises.
Therefore, the train's state and the linked balises are retrieved for the first time from the topic, which together with the decision making leads to a high computational overhead.
The third and fourth peak (after nine and eleven seconds) are again due to an evaluated balise telegram.
However, only the updated train state is retrieved.

The seconds scenario \texttt{Unlinked Balise} (\textbf{S2}) starts after 15 seconds.
The peaks, again, can be explained by data publication and retrieval.
At 23 seconds after the measurements started, an unlinked balise is encountered and the train needs to stop.
This leads to an increase in \abr{CPU} utilization for both the leader and the followers.

The third scenario \texttt{Balise not where expected} (\textbf{S3}) starts at 26 seconds.
At 35 seconds after the measurements were started, an unexpected balise has been encountered, which can be seen in the graph as well.

\subsubsection{Message Exchanges}
\todo{Mention that no spare activation happens in the measurements}
\todo{Mention that Hauptlast is on leader (see total received and sended)}
As the scenarios are run, messages are exchanged between the replicas via the different \abr{DDS} topics.
For the message exchange evaluation, the leader has been manually stopped after the second scenario and before the third scenario is run.
The third scenario has been run with only a leader and one follower.
In the following, the number of messages that occur during the simulation of the three successive scenarios, that were described above, is analyzed.
At first, the sent messages are examined.
Therefore, the total number of sent messages, as well as the number of messages for the consensus topics and the state topics are analyzed.
Afterwards, the same is done for the received messages.
Besides the topics named above, the \texttt{Input} topic will be taken into account for the received messages.

For each measurement, the heartbeat timer has been set to 100000µs so that ten messages per second can be attributed to heartbeat messages.

\paragraph{Total Messages Sent}

\begin{figure}[!hb]
	\centering
	\includegraphics[width=0.75\linewidth]{images/plots/TotalMessagesSent}
	\caption{Overview about sent messages in the system. A distinction is made between the messages that are send by the system's leader and the followers . In addition, the overall number of sent messages is shown.}
	\label{fig:PlotTotalMessagesSent}
\end{figure}

The total number of messages that were sent by replicas in the system per seconds are shown in~\autoref{fig:PlotTotalMessagesSent}.
Beginning and end of each of the three scenarios can be seen in the plot because the total number of sent messages increases to 20-25 messages per second while driving and reduces to twelve messages per second when the train stands still.
A majority of messages is sent by the system's leader of which ten alone are due to heartbeat messages.
The rest accrue from cluster-wide monitoring of the braking curve.
While the train is driving, it is the leader's responsibility to update the system's global state which explains the big increase in sent messages during journey.

\paragraph{Consensus Messages Sent}

\begin{figure}[!hb]
	\centering
	\includegraphics[width=0.75\linewidth]{images/plots/ConsensusMessagesSent}
	\caption{The sent messages for all topics used for the consensus algorithm. \texttt{AppendEntries} is used for heartbeat messages and for log replication. \texttt{AppendEntriesReply} functions as a way to transmit decisions from followers to the leader. Via \texttt{RequestVote} and \texttt{RequestVoteReply}, the leader voting happens. After 25 seconds, the previous leader crashes, which results in a drop of heartbeat messages in the \texttt{AppendEntries} topic. After a new leader is elected, the amount of heartbeat messages goes back to the previous one.}
	\label{fig:PlotConsensusMessagesSent}
\end{figure}

Consensus topics include \texttt{AppendEntries}, \texttt{AppendEntriesReply}, \texttt{RequestVote}, and \texttt{RequestVoteReply}.
The number of messages published to the consensus topics are depicted in~\autoref{fig:PlotConsensusMessagesSent}.
The majority of messages gets published to the \texttt{AppendEntries} topic, because the leader periodically published ten heartbeat messages per second.
It can be seen that a leader election takes place at the beginning, because a vote request and two vote replies are sent.
Another leader election happens after the second scenario has finished after 25 seconds.
The fact that the previous leader crashed is reflected in the collapse of heartbeat messages on the \texttt{AppendEntries} topic.
Again, a vote request is published to the corresponding topic by the first replica that notices the absent leader.
This time, only one vote reply is published because only one other replica remains in the system.
The new leader establishes itself afterwards because the amount of heartbeat messages goes back to the number it was before the previous leader crashed.


\paragraph{State Messages Sent}

\begin{figure}[!hb]
	\centering
	\includegraphics[width=0.75\linewidth]{images/plots/StateMessagesSent}
	\caption{The sent messages for the topics that store the system's global state. It can be seen that, at the beginning of each scenario, a new \glsentryfull{MA} and linked balises are stored. The train's position is simulated and updated every 100ms in the \texttt{TrainState} topic.}
	\label{fig:PlotStateMessagesSent}
\end{figure}

State topics include \texttt{TrainState}, \texttt{MovementAuthority}, and \texttt{LinkedBalises}.
The amount of messages published to these topics can be seen in~\autoref{fig:PlotStateMessagesSent}.
These messages are all sent by the leader, because it is the only replica that is allowed to publish to these topics.
It is noticeable that most state messages are published on the \texttt{TrainState} topic.
This is because the position is simulated every 100000µs and written to \texttt{TrainState}.
The small peaks for the \texttt{TrainState} topic indicate that the train's position is reset when the train encounters a linked balise.
Messages to \texttt{MovementAuthority} and \texttt{LinkedBalises} are only published at the beginning of each scenario, because at this time the \abr{MA} and linked balises are communicated to the system.
It can futher easily be seen that three balises are linked in the first and the third scenario but only two linked balises are added for the second.

\paragraph{Total Messages Received}

\begin{figure}[!hb]
	\centering
	\includegraphics[width=0.75\linewidth]{images/plots/TotalMessagesReceive}
	\caption{Overview about all messages that are received from replicas within the system.}
	\label{fig:PlotTotalMessagesReceive}
\end{figure}

The total number of messages received through \abr{DDS} topics can be seen in~\autoref{fig:PlotTotalMessagesReceive}.
What is interesting is, that the leader receives and reads more messages than the two followers combined.
This is mostly because the leader is responsible for reading the input topic and because it receives and reads messages from both followers.
Another interesting circumstance can be seen after 25 seconds - which is when the previous leader crashes - since the amount of messages that the followers received is halved.
This is because only one, instead of two, followers remain in the system.

\paragraph{Input Messages Received}

\begin{figure}[!hb]
	\centering
	\includegraphics[width=0.75\linewidth]{images/plots/InputMessagesReceive}
	\caption{The number of input messages received by the system per second. Based on the distribution the scenarios procedure can be seen. At the beginning, two input messages contain the \glsentryfull{MA} and linked balises. One second later, the first balise telegram is received. The last two balise telegrams follow each other at an interval of two seconds. The distance between the last two balises is greater for the last scenario. This is because the last scenario, it is simulated that the balise is not at the position that was specified in the linking phase.}
	\label{fig:PlotInputMessagesReceive}
\end{figure}

The individual scenarios' structures can be traced by the input messages that the sytem receives.
This is depicted in~\autoref{fig:PlotInputMessagesReceive}.
At the beginning of each scenario, one message with the \abr{MA} and one with the linked balises are sent.
Because the train's speed is simulated to be constant, the intervals between the incoming balises telegrams are the equal for the first and the second scenario.
In the third scenario it is simulated that the last balise is at a different position than specified in the linking phase.
Therefore, the distance between the last two balises is greater in the third scenario.
It can also be seen that although after 25 seconds the old leader crashes, the system does not miss any input message.
However, unlike the first two scenarios, the input messages are buffered and therefore processed all at once, which is why the peak is higher.

\paragraph{Consensus Messages Received}

\begin{figure}[!hb]
	\centering
	\includegraphics[width=0.75\linewidth]{images/plots/ConsensusMessagesReceive}
	\caption{All messages that are received by a component on the topics used for finding a consensus. After 25 seconds, a new leader is elected and only one follower remains in the system. Therefore, messages are read from \texttt{RequestVote} and \texttt{RequestVoteReply}. Further, since the number of followers got reduced by half, also the number of received heartbeat messages got reduced.}
	\label{fig:PlotConsensusMessagesReceive}
\end{figure}

The number of received messages on the consensus topics is shown in~\autoref{fig:PlotConsensusMessagesReceive}.
It is noticeable that after 25 seconds the number of AppendEntries messages received drops.
Further, a message is received on the \texttt{RequestVote} and on the \texttt{RequestVoteReply} topic, respectively.
From this it can be seen that a leader election occurred.
Because there is only one follower left in the system, the number of received heartbeat messages is halved.

\section{\glsentrylong{DDS} Evaluation}
\abr{DCPS} systems, such as \abr{DDS}, are designed for machine-to-machine communication in real-time applications~\cite{omgDDSspec}.
It builds upon a decoupling from senders and receivers to facilitate a flexible and scalable system.
However, decoupling of senders and receivers is not always possible in a consensus-based and safety-critical redundant system because a follower replica that receives heartbeat messages can become sender of heartbeat messages at any time when the previous leader becomes unavailable.

Further, \abr{DDS} is designed as a standard for distributed application communication and integration and therefore implements variety of features.
Not all of these features are required for a system like the one that is assessed in this work.
\\

Therefore in this section, a feature subset of ADLINK's \texttt{OpenSplice DDS} is presented that is sufficient to implement the requirements placed on the system.
Afterwards, the suitability of \abr{DDS} for safety-critical and consensus-based redundant systems is discussed.
\subsection{\glsentrylong{DDS} Subset Identification}

For reducing costs incurred in the system approving process, it is important to limit the amount of utilized features and lines of code.
Therefore, a subset of \abr{DDS} features from OpenSplice DDS, that is sufficient for building a redundant system, is identified.
An overview about this subset is given and justified in the following.
This overview only contains the methods and functionalities that are directly used by the application, the middleware, however, might utilize other functionalities to implement the features mentioned in this section.

\paragraph{Domain Module}
A \abr{DDS} application's main building blocks comprise a \texttt{Domain}, one or multiple \texttt{DomainParticipants}, as well as a \texttt{DomainParticipantFactory} for creating new \texttt{DomainParticipants}~\cite{omgDDSspec}.
Besides these classes, the domain module contains various functionalities, from which only the following subset is used within the exemplary implementation.

\begin{itemize}
\item \textit{DDS\_Entity\_get\_instance\_handle} Is used for acquiring an instance handle to mark it as processed in the leader's commit phase.
\item \textit{DDS\_DomainParticipant\_create\_(publisher|subscriber|topic)} Is used for creating \abr{DDS} publishers, subscribers or topics.
\item \textit{DDS\_DomainParticipant\_delete\_(publisher|subscriber|topic)} Is used for deleting \abr{DDS} publishers, subscribers or topics.
\item \textit{DDS\_DomainParticipantFactory\_get\_instance} Used for acquiring the \texttt{DomainParticipantFactory} singleton.
\item \textit{DDS\_DomainParticipantFactory\_(create|delete)\_participant} Used for creating a new participant in the \abr{DDS} domain.
\end{itemize}

Even though \textit{DDS\_DomainParticipant\_get\_default\_(publisher|subscriber|topic)\_qos} is used in the implementation, it is not mandatory because one could set the default \abr{QOS} settings directly on the corresponding entity.

\paragraph{Topic-Definition Module}
The topic definition module contains everything related to topics.
The exemplary implementation utilizes \texttt{Topics}, for describing data in the application, and \texttt{TypeSupport} which describes a data type that is bound to a \texttt{Topic}.
Apart from these two classes, the following methods are used:

\begin{itemize}
\item \textit{DDS\_TypeSupport\_\_alloc} Used for creating a new \texttt{TypeSupport}.
\item \textit{DDS\_TypeSupport\_get\_type\_name} Used for getting a data type's default name, which is required to register it later on.
\item \textit{DDS\_TypeSupport\_register\_type} Used for registering a new data type name on a \texttt{DomainParticipant}.
\end{itemize}


\paragraph{Publication Module}
\texttt{Publishers}, as well as \texttt{DataWriters}, which are used for data distribution, are located within the publication module.
Besides these two classes, the following functionality is required for the exemplary implementation:

\begin{itemize}
\item \textit{DDS\_Publisher\_(create|delete)\_datawriter} Used for creating, respectively deleting, \texttt{DataWriter} objects.
\item \textit{DDS\_DataWriter\_write} Used for writing new data to an instance on a \abr{DDS} topic.
\item \textit{DDS\_DataWriter\_dispose} Used for marking processed inputs for deletion so that they are not processed twice.
\end{itemize}

\textit{DDS\_Publisher\_copy\_from\_topic\_qos} and \textit{DDS\_Publisher\_get\_default\_datawriter\_qos} are not necessarily required due to the same reasons stated above.

\paragraph{Subscription Module}
The subscription module resembles the publication module, but for reading and receiving data.
However, besides \texttt{Subscribers} and \texttt{DataReaders}, it also contains \texttt{DataSample}, \texttt{SampleInfo}, \texttt{ReadCondition}, and \texttt{QueryCondition}, which are utilized in the exemplary implementation.
\texttt{Subscribers} and \texttt{DataReaders} are required for receiving data, which is represented by a \texttt{DataSample}.
The \texttt{SampleInfo} data structure contains additional information for each \texttt{DataSample}, such as its instance handle or state, which makes it mandatory for the implementation as well.
\texttt{ReadConditions} are used in combination with \texttt{WaitSets}, so that the application can react to certain middleware states.
\texttt{QueryConditions} are extensions for \texttt{ReadConditions} that allow to filter the data samples that are covered by a \texttt{ReadCondition}.
All filter operations could be implemented directly in the application code, which would make \texttt{QueryConditions} redundant.

Further, the following functionalities from the subscription module were used:

\begin{itemize}
\item \textit{DDS\_Subscriber\_(create|delete)\_datareader} Used for creating, respectively deleting, \texttt{DataReaders}.
\item \textit{DDS\_DataReader\_(create|delete)\_readcondition} Used for creating, respectively deleting, \texttt{ReadConditions}.
\item \textit{DDS\_DataReader\_create\_querycondition} Used for creating \texttt{QueryConditions}.
\item \textit{DDS\_DataReader\_take\_w\_condition} Used for reading and removing data from a \texttt{DataReader}, filtered by a \texttt{ReadCondition}.
\item \textit{DDS\_DataReader\_return\_loan} Required for indicating the middleware that the application finished accessing a sequence of data. This allows the middleware to work with the data again.
\end{itemize}

\paragraph{Infrastructure Module}
The classes that are used from the infrastructure module include \texttt{QosPolicy} and \texttt{WaitSet}.
The \texttt{QosPolicy} class implements \abr{DDS}'s \abr{QOS} functionalities and is therefore mandatory for the system's correctness.
\texttt{WaitSets} are applied for ensuring the application's compliance with its time restrictions, because they allow the assignment of timeouts.
In addition to that, the following functionalities from the infrastructure module are applied:

\begin{itemize}
\item \textit{DDS\_WaitSet\_\_alloc} Used for creating new \texttt{WaitSets}.
\item \textit{DDS\_WaitSet\_attach\_condition} Used to attach a condition to the corresponding \texttt{WaitSet}.
\item \textit{DDS\_WaitSet\_detach\_condition} Used to detach a condition from the corresponding \texttt{WaitSet}.
\item \textit{DDS\_WaitSet\_wait} Stops the according thread until at least one of the attached conditions evaluates to true.
\end{itemize}

Further, the following \abr{QOS} are applied in the application:
\todo{Add after deciding}

\subsection{Challenges of \glsentrylong{DDS} in Consensus-based Redundant Systems}
In this section, the suitability of \abr{DCPS} systems, like \abr{DDS}, for consensus based redundant systems is discussed.

\paragraph{Advantages}
On the one hand, \abr{DDS} allows to utilize \abr{QOS} settings to specify what is expected from the system rather than how it should be done.
Through \texttt{ReadConditions} and \texttt{QueryConditions}, messages can be filtered.
Further, \texttt{WaitSets} and \abr{QOS}-settings can be used to declare time restriction and requirements.
This facilitates a reliable and predictable machine-to-machine communication.

In addition, the middleware supports automatic discovery of replicas to that the number of used replicas is dynamically scalable.
For example, an active spare unit can be added to the system at any time and is automatically discovered and included.
This would also allow to apply a three-out-of-five system rather than a two-out-of-three system and use the same software.

Finally, \abr{DDS} supports fast transmission times and a manageable hardware utilization, as shown in measurements.

\paragraph{Challenges}
On the other hand, the \abr{DDS} standard has been designed around a loose coupling of publishing and subscribing components.
This is not possible in dynamic systems like the system build in this work.
Because the applied \abr{TMR} builds upon \texttt{Raft}'s leader and log replication concepts, any replica needs to be prepared to become leader or follower at any time.
For example, the system's leader is responsible for publishing heartbeat messages to the \texttt{AppendEntries} topic, while a follower needs to subscribe and receive these heartbeats.
Therefore, replicas are required to dynamically alternate between sending and receiving data.
One way to solve this challenge is to create and delete objects for publishing and subscribing to topics dynamically.
However, this can lead to a sudden allocation and deallocation of dynamic memory which can lead to unpredictable time performances.
Thus, every replica allocates the objects that are required to publish and subscribe to every topic at program start, which leads to other challenges.

When each replica is subscribed to a topic that applies a \texttt{History}- and \texttt{ResourceLimit} \abr{QOS}, the messages that are received by \texttt{DataReader} objects must be taken by the application.
Hence, a leader that is subscribed to the \texttt{AppendEntries} topic in preparation of becoming a follower must take all heartbeat messages from the topic that were published by itself.
This introduces a computational overhead, though a small one as the system's resource utilization in~\autoref{fig:PlotCPUUsageIdleTime} show.

Lastly, a system that utilizes \abr{DDS} for communication needs to support concurrent computations when more that one message should be processable at any time.
This introduces potential race conditions or deadlocks and can complicate the development and approval process.
Since the developed system, for example, needs to react to vote requests and log replication messages at any time, a concurrent system design is unavoidable.
As discussed in~\autoref{subsub:raceConditions} though, the implemented system applies a race condition free concurrent algorithm.

\iffalse
\paragraph{State Messages Received}

\begin{figure}[!hb]
	\centering
	\includegraphics[width=0.75\linewidth]{images/plots/StateMessagesReceive}
	\caption{}
	\label{fig:PlotStateMessagesReceive}
\end{figure}

\todo{Do measurement with killing leader, but this time activate spare and show a topic selection that hot standby works}

Do experiement and structure in way such as SakicTimeInConsensus

Leader stable for 45 minutes with 500000x07x2

For the tranmit time, I measured 20 messages each time and took the average (calculate standard deviation)
\fi

    
	% Bibliographie
	\ifisbook\cleardoubleemptypage\fi
	\phantomsection\addcontentsline{toc}{chapter}{\refname}
	\printbibliography[category=cited]

	% ggf. Anhang
	% \appendix\include{content/appendix} % example

	% ggf. bei englischen Arbeiten den deutschen Abstract nach hinten verschieben
	% \ifisbook\pagestyle{plain}\cleardoubleemptypage\include{content/abstract_deu}\fi

	% Eigenständigkeitserklärung
	% \ifisbook\pagestyle{plain}\cleardoubleemptypage\include{content/disclaimer}\fi

\end{document}
