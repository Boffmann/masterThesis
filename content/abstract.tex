% => Wenn die Arbeit auf Deutsch verfasst wurde, verlangt das Studienreferat KEINEN englischen Abstract

% % englischer Abstract
\null\vfil
%\begin{otherlanguage}{english}
\begin{center}\textsf{\textbf{\abstractname}}\end{center}
%
\noindent \glsentryfull{ETCS} on-board units are safety-critical systems whose reliability plays a vital role in the integrity of railway operation.
While redundancy is a typical method for increasing fault tolerance, reliability, and safety, it also adds a communication and computation overhead.
Further, the development and maintenance of highly safety-critical applications are resource-intensive.
This thesis provides background and implementation of a redundant and fault-tolerant ETCS on-board system using real-time \glsentryfull{DCPS} machine-to-machine communication and consensus-based voting.
Besides behavioral concepts, such as a leader election and decision-making algorithm, the work exposes global system state and system recovery mechanisms in distributed systems.
The implementation's functionality, safety, and reliability are evaluated based on a subset of ETCS in a simulated environment.
The results confirm DCPS concepts applicable for solving the communication and computation overhead of distributed and redundant computation in a real-time and safety-critical environment.
Furthermore, the findings also apply to an architecture of general-purpose Programmable Logic Controllers (PLCs).
Thereby, this work facilitates the development of safe and more cost-efficient on-board systems for future ETCS applications.

%\end{otherlanguage}
\vfil\null

% => Wenn die Arbeit auf Englisch verfasst wurde, verlangt das Studienreferat einen englischen UND deutschen Abstract (der dt. Abstract kann dann ggf. auch ans Ende der Arbeit)

% deutsche Zusammenfassung
\null\vfil
\begin{otherlanguage}{ngerman}
\begin{center}\textsf{\textbf{\abstractname}}\end{center}

\noindent 

Die Fahrzeugsysteme des Europäischen Zugbeeinflussungssystems (ETCS) sind sicherheitskritische Systeme, deren Zuverlässigkeit eine Entscheidende Rolle bei der Betriebssicherheit der Bahn spielt.
Um die Fehlertoleranz, die Zuverlässigkeit und die Sicherheit von Systemen zu erhöhen, werden üblicherweise redundante Informationen und Berechnungen verwendet.
Diese führen allerdings auch zu einem erhöhten Kommunikations- und Bearbeitungsaufwand.
Darüber hinaus sind Entwicklung und Instandhaltung von sicherheitskritischen Anwendungen ressourcenintensiv.
In dieser Arbeit werden Verfahren und Implementierung eines verteilten, redundanten und fehlertoleranten ETCS-Fahrzeugsystems vorgestellt.
Die systeminterne Kommunikation basiert auf einer echtzeitfähigen und datenzentralen Zwischenanwendung, die dem Publish-Subscribe Muster folgt (DCPS).
Weiterhin werden redundate Zwischenergebnisse über ein konsensbasiertes Abstimmungsverfahren dedupliziert.
Neben Verhaltenskonzepten, wie der Wahl eines Systemanführeres und einem Entscheidungsfindungsalgorithmus, stellt diese Arbeit Möglichkeiten für einen globalen Systemzustand und zur Erhohlung von Fehlern in verteilten Systemen vor.
Die Funktionalität, Sicherheit und Zuverlässigkeit der Implementierung werden, basierend auf einer Teilmenge von ETCS, in einer simulierten Umgebung evaluiert.
Die Ergebnisse bestätigen die Anwendbarkeit von DCPS-Konzepten zur Lösung des Kommunikations- und Rechenaufwandes verteilter und redundanter Systeme in echtzeit- und sicherheitskritischen Umgebungen.
Darüber hinaus zeigt sich, dass die Erkenntnisse auch für eine Architektur von universell einsetzbaren Speicherprogrammierbaren Steuerungenn (SPS) gelten. 
Dadurch erleichtert diese Arbeit die Entwicklung von sicheren und kostengünstigeren ETCS-Fahrzeugsystemen.

\end{otherlanguage}
\vfil\null


