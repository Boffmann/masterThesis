% => Wenn die Arbeit auf Deutsch verfasst wurde, verlangt das Studienreferat KEINEN englischen Abstract

% % englischer Abstract
\null\vfil
%\begin{otherlanguage}{english}
\begin{center}\textsf{\textbf{\abstractname}}\end{center}
%
\noindent \glsentryfull{ETCS} on-board units are safety-critical systems whose reliability plays a vital role in the integrity of railway operation.
While redundancy is a typical method for increasing fault tolerance and hence reliability and safety, it also elevates synchronization and runtime complexity.
This thesis proposes the usage of DDS - a real-time capable machine-to-machine communication standard following the \glsentryfull{DCPS} pattern - to deal with the added complexity.
It further provides background and implementation of a redundant and fault-tolerant ETCS on-board system using real-time DCPS communication and consensus-based voting.
Besides data-centric implementations of behavioral concepts, such as a leader election and decision-making algorithm, the work contributes practical solutions to global state and recovery subjects in distributed systems.
Functionality, safety, and reliability of the implementation are evaluated using a subset of ETCS and a cluster of four Raspberry Pi-based PLCs within a simulated environment.
The results show that (1) the redundant system facilitates real-time computation with high network throughput (up to 77 Mbit/s) and fast consensus-building ($\mu$=$6.5$~ms;$~\sigma$=$1.3$~ms). (2) Consensus-based DCPS communication increases reliability and enables fast system recovery ($\mu$=$5.47$~ms;$~\sigma$=$0.77$~ms upon recognizing a fault). (3) A subset of DDS is already sufficient to accomplish reliability and safety.
Thereby, this work facilitates the development of safe and cost-efficient on-board systems for future ETCS applications.

%\end{otherlanguage}
\vfil\null
\clearpage
% => Wenn die Arbeit auf Englisch verfasst wurde, verlangt das Studienreferat einen englischen UND deutschen Abstract (der dt. Abstract kann dann ggf. auch ans Ende der Arbeit)

% deutsche Zusammenfassung
\null\vfil
\begin{otherlanguage}{ngerman}
\begin{center}\textsf{\textbf{\abstractname}}\end{center}

\noindent 
\todo{Translate new abstract to german}
Die Fahrzeugsysteme des Europäischen Zugbeeinflussungssystems (ETCS) sind sicherheitskritische Systeme, deren Zuverlässigkeit eine entscheidende Rolle bei der Betriebssicherheit der Bahn spielt.
Um die Fehlertoleranz, die Zuverlässigkeit und die Sicherheit von Systemen zu erhöhen, werden üblicherweise redundante Informationen und Berechnungen verwendet.
Diese führen allerdings auch zu einem erhöhten Kommunikations- und Bearbeitungsaufwand.
Darüber hinaus sind Entwicklung und Instandhaltung von sicherheitskritischen Anwendungen ressourcenintensiv.
In dieser Arbeit werden Grundlagen, Funktionsweisen und die Implementierung eines verteilten, redundanten und fehlertoleranten ETCS-Fahrzeugsystems vorgestellt.
Die systeminterne Kommunikation basiert auf einer echtzeitfähigen Zwischenanwendung, die dem datenzentralen Publish-Subscribe (DCPS) Muster folgt.
Weiterhin werden redundante Zwischenergebnisse über ein konsensbasiertes Abstimmungsverfahren dedupliziert.
Neben Verhaltenskonzepten, wie der Wahl eines Systemanführers und einem Entscheidungsfindungsalgorithmus, stellt diese Arbeit Möglichkeiten für einen globalen Systemzustand und zur Erholung von Fehlern in verteilten Systemen vor.
Die Evaluation von Funktionalität, Sicherheit und Zuverlässigkeit der Implementierung erfolgt in einer simulierten Umgebung, basierend auf einer Teilmenge von ETCS.
Die Ergebnisse bestätigen die Anwendbarkeit von DCPS-Konzepten zur Lösung des Kommunikations- und Rechenaufwandes verteilter und redundanter Systeme in echtzeit- und sicherheitskritischen Umgebungen.
Darüber hinaus zeigt sich, dass die Erkenntnisse auch für eine Architektur von universell einsetzbaren Speicherprogrammierbaren Steuerungen (SPS) gelten. 
Dadurch erleichtert diese Arbeit die zukünftige Entwicklung von sicheren und kostengünstigen ETCS-Fahrzeugsystemen.

\end{otherlanguage}
\vfil\null


