\chapter{Conclusion and Future Work}

\section{Conclusion}
This thesis describes the implementation of a fault-tolerant redundant architecture that utilizes \abr{DCPS} communication for an \abr{ETCS} on-board unit on \abrpl{PLC}.

In the first part, a suitable redundant architecture was established in three steps.
First, typical redundancy techniques for fault-tolerant designs were listed and classified.
Second, an \abr{ETCS} subset and scenario were identified.
The scenario provides the basis for a fault model, a concrete system architecture, and the subsequent implementation.
Third, a consensus-based and redundant approach with a hot standby was identified as the most appropriate system architecture solving challenges of \abr{TMR}.
On the one hand, it prevents the voter from being a single point of failure.
On the other hand, it provides higher reliability, flexibility, and scaling through \abr{DDS}.

In the second part, the thesis describes the implementation of the redundant architecture using a combination of voting and consensus.
\texttt{Raft} was chosen as the basis for the consensus part.
Besides a leader election and decision-making algorithm, methods for system recovery that increase reliability are presented.
Finally, the implementation is evaluated using a simulator and automatic integration tests.
Results show fault-tolerance in the event of individual component failure, as well as sufficient real-time capability and system recovery.
Based on these results, a subset of \abr{DDS} functionality was assembled.
This facilitates the creation of specialized \abr{DDS} middleware so that the costs for developing and approving future \abr{ETCS} on-board systems can be greatly reduced.



\section{Future Work}
