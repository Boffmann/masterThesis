\chapter{Conclusion and Future Work}

\section{Conclusion}
This thesis described the implementation of a fault-tolerant redundant architecture that utilizes \abr{DCPS} communication for an \abr{ETCS} on-board unit on general-purpose \abrpl{PLC}.

In the first part, a suitable redundant architecture was established in three steps.
First, typical redundancy techniques for fault-tolerant designs were listed and classified.
Second, an \abr{ETCS} subset and scenario were identified.
The scenario provides the basis for a fault model, a concrete system architecture, and the subsequent implementation.
Third, a consensus-based and redundant approach with a hot standby was identified as the most appropriate system architecture solving challenges of \abr{TMR}.
On the one hand, it prevents the voter from being a single point of failure.
On the other hand, it provides higher reliability, flexibility, and scaling through \abr{DDS}.

In the second part, the thesis describes the implementation of the redundant architecture using a combination of voting and consensus.
\texttt{Raft} was chosen as the basis for the consensus part.
Besides a leader election and decision-making algorithm, methods for system recovery that increase reliability are presented.
Finally, the implementation is evaluated using a simulator and automatic integration tests.
Results verify fault-tolerance in the event of individual component failure, as well as sufficient real-time capability and system recovery.
Based on these results, a subset of \abr{DDS} functionality was assembled.
This facilitates the creation of specialized \abr{DDS} middleware so that the costs for developing and approving future \abr{ETCS} on-board systems can be greatly reduced.


\section{Future Work}

The system developed in this thesis bases on the fault model of an \abr{ETCS} subset.
As a result, the underlying fault model is limited - compared to the complete \abr{ETCS} specification.
For example, instead of track-side balise groups, individual balises were assumed.
Thereby, consideration of the balise group's alignment drops from the fault model.
Additionally, the speed and gradient profiles of \abrpl{MA} were not considered, which affects braking curve supervision.
Consequently, the fault model mainly focused on the safety and real-time capability of input processing.

Future work concerns considerations of the entire ETCS specification and thus an extended fault model.
Based on this, other test scenarios can be added to the simulator.
Further, advanced checks not related to incoming messages - such as detecting missed balises - can be implemented.
\\

The results of this work are based on some assumptions.
For example, it is assumed that messages are processed correctly by the middleware, and no data is lost or corrupted during transmission.
Since this is not always guaranteed, other redundancy concepts, such as information and time redundancies, could be applied.
In the context of this, redundant and diverse transmission channels, such as CAN-Bus or FlexRay, can be used, or the same data can be sent multiple times.
Further, \abr{ECC} concepts, such as Hamming codes or Reed-Solomon-Codes~\cite{ReedSolomonCodes}, could be applied to detect and correct transmission errors.
\\

This work's fault model omits Byzantine faults.
Reliable decision-making, despite one manipulative component, requires at least four voters and two rounds of communication~\cite{GamerIncreasingMOON}.
Thereby, the network's delay and load increase.
For use in highly safety-critical applications, Byzantine faults should also be covered, and the resulting impact on the system must be investigated.
Further, diverse hardware redundancy can be applied to reduce the chance of having multiple manipulative components and common core errors.

In addition, the system is intended as an \abr{ETCS}1 on-board unit.
As such, it interacts with the train's hardware, which has not yet been covered within this work.

The train hardware includes, among other things, the brakes, as well as position sensors.
Furthermore, in practice, the inputs are not transmitted directly via \abr{DDS} topics but by antenna systems.
Finally, the internal consistency of the components can be monitored.
Concepts of self-checking hardware exist, as described by Wakerly~\cite{SelfCheckingProcessorDesign}.
Based on the results from Bijsma \etal, \abr{DDS} can be applied together with component checking methods to extend the system's distributed safety mechanisms~\cite{DistributedSafety2020}.