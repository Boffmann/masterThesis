\chapter{Introduction}
	
Railway operation is undergoing a rapid change.
Ongoing developments in computation and railway applications have improved interlocking technologies, train length and travel speed.
Further, with the introduction of new communication technologies, such as 5G, new possibilties of transmitting information have emerged.
With improvements in railway technologies, the significance of \gls*{ATO} systems to monitor and control a train's operating procedures rises~\cite{YIN2017RNDofATO}.
This has not only the potential to decrease running expenses, but could also help to disburden drivers~\todo{ref}.
\\

Nowadays, multiple \gls*{ATO} systems have been developed in different countries.
In order to solve the thereby caused incompatabilities among the different \gls*{ATO} systems, \gls*{ETCS}, as a part of \gls*{ERTMS}, has been established as an international standard for \gls*{ATO} systems~\cite{ETCS26}.
Because of the railway domain being a safety-critical environment, a system solving the \gls*{ETCS} specification must comply with certification requirements, such as safety and reliability characteristics, to be applicable for practical use.

The compliance of system requirements is ensured during an approval process.
The thereby applied verification, validation and certification steps are very time- and cost-intensive.
Moreover, the approval process needs to be repleated every time the system updates.
\\

A solution for simplifying the approval process is to combine related functions into modules and map these modules onto a distributed system.
A benefit of this concept is, that it allows to build safe and reliable systems out of unsafe and unreliable parts (ref slides fault tolerance).
The distribution approach is already applied in modern safety-critical systems, but also introduces new challenges such handling the communication overhead and dealing with node-, network and computation failures~\cite{DistributedSafety2020}.

\section{Contribution}

In this work, an approach for building a distributed \gls*{ATO} system that meets the certification requirements for \gls*{ETCS} is presented.
The key idea is to use proven architecture patterns, such as redundancy, system self-testing and fault detection, to build a safe and reliable system out of unreliable parts.
Further, for communication in the distributed system, the applicability of \gls*{DDS}, a dependable machine-to-machine communication standard following the publish-subscribe pattern, is analyzed.
The worked out approach is finally demonstrated based on \gls*{ETCS} balise telegram evaluation.
Based on my work, future ETCS implementations can be realized.

\section{Outline}

\paragraph{\autoref{chptr:system_requirements}}
In this chapter, a set of requirements for a \gls*{ATO} system is established.
For quantitative requirements, such as safety or reliability, mathematical concepts for evaluation are presented.

\paragraph{Concepts}
In this chapter, different possible system architecture candidates are presented.
Further, these candidate's appliance with the system requirements from~\autoref{chptr:system_requirements} are assessed.
Therefore, a high level analysis towards reliability and safety is made.
The result of this chapter is a suitable system architecture candidate, as well as specific hardware and software specifications to meet the architecture's requirements.

\paragraph{Implementation}
In this chapter the system implementation is described.

\paragraph{Evaluation}
Here, the practical system is evaluated and a conclusion is made.
