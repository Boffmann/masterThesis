\chapter{Introduction}

%https://www.myperfectwords.com/blog/thesis-writing/writing-a-thesis-introduction

% Attention grabbing hook statement
Railway operation is undergoing rapid changes.
%
% General introduction to topic through general statements
Contant improvements in computation, communication technologies and railway applications laid the foundation for \abr{ATO} systems.

%- Mention that ATO whould diminish human errors (find statistic of how often human errors lead to catastropic failures)
At the same time, the standardization of signalling and control components in the \abr{ETCS} facilitated train operation across borders and heightened the significance of \abr{ATO} systems~\cite{YIN2017RNDofATO}.

% Describe problem statement
However, with the railway being a safety-critical system, any component that is applied in the \abr{ETCS} context must comply with strict certification requirements to ensure its safety and reliability.
The compliance of system requirements is ensured during an approval process.
The thereby applied verification, validation and certification steps are time- and cost-intensive, which adds to the already high development costs.
Moreover, the approval process needs to be repeated every time the system gets updated.
Increasing costs for \abr{ETCS} systems could lead to a situation where it is not profitable to equip certain regions, connections, or train models with \abr{ETCS}.
\\

A typical way to reduce a system's complexity and thereby conserve implementation and approval costs is by distributing the computation to multiple execution units.
However, distributed computation also introduces new challenges such as coping with the communication overhead and dealing with node-, network-, and computation faults~\cite{DistributedSafety2020}.

% Main hypothesis
On the one hand, redundant computation is a key technique for increasing a system's safety~\cite{TanenbaumSteen07}.
Generally speaking, redundancy entails additional resources, that are not required for a system's functional operation, but adds certain characteristics such as fault detection and fault tolerance~\cite{BarryFaultToleranceAnalysis}.
Various well-established redundancy patterns have been proven in practice, which include hardware-, software-, information-, and time redundancy.
In contrast to non-redundant systems, redundancy allows to build safe systems out of less safe and cheaper parts.

On the other hand, the \abr{DDS} standard, which is a \abr{DCPS} for machine-to-machine communication, can be applied for facilitating safe and real-time communication among distributed execution units~\cite{DistributedSafety2020}.
The standard, which is intended to be implemented as a middleware, allows the specification of \abr{QOS} to specify the service's behavior in a declarative way.
Through the \abrpl{QOS}, the communication can be defined to be reliable and real-time.
\\
%Redundancy further allows to build safe systems out of less safe and cheaper parts, as opposed to non-redundant systems.
%\\

In this work, it is investigated whether the \abr{DDS} standard is applicable for building redundant, distributed, real-time, and safety-critical systems.
Further, a suitable minimal subset of \abr{DDS} features for achieving the desired properties, is determined.
By finding a minimal subset, the amount of aspects to verify can be decreased and approval costs can be further reduced.
Therefore, a redundant system, that comprises of four homogeneous and interconnected \textit{Revolution Pi} from Kunbus \todo{cite}, is build and assessed.
The \textit{Revolution Pi} is a robust \abrpl{PLC} that is build upon a \textit{Raspberry Pi} and features a real-time capable operation system.
On the middleware layer, OpenSplice \abr{DDS} from ADLINK \todo{cite} is applied, whose applicability for safety-critical railway applications has been proven~\cite{SchmidtMissionCriticalChallenges}.
\todo{Menion results and main realization}

The further work is structured as follows:

\paragraph{\autoref{chptr:concepts}}
In this chapter, an overview about popular redundancy patterns and characteristics for building safe and reliable systems, as well as \abr{ETCS} and \abr{DDS}, is given.
Afterwards, mathematical concepts for evaluating the safety and reliability of these systems are pointed out.
Finally, a selection of related work that uses \gls{DDS} in safety critical applications, is presented and set in relation to this work.

\paragraph{\autoref{chptr:redundantSystemsCompare}}
Key building blocks for designing distributed systems with \abr{DDS} are described.
Afterwards, the redundancy patterns that were presented in~\autoref{chptr:concepts} are opposed and evaluated.
Further, exemplary concepts and architectures for each pattern are described and assessed based on their safety and realizability with \abr{DDS} concepts.
Therefore, not only voting-based concepts, but also a way of building consensus driven redundant systems, are described.
Finally, a mixture of consensus- and voting-based approaches is identified as the most suitable concept for satisfying the desired system's requirements.

\paragraph{\autoref{cpt:Implementation}}
In this chapter, an exemplary implementation of a consensus algorithm, that follows the concepts of \texttt{Raft}~\cite{RaftConsensusPaper}, is presented.
The implementation facililtates redundant computation and applies \abr{DDS} concepts for synchronizing the utilized execution units.
In addition, the system's applicability is demonstrated based on an \abr{ETCS} use-case.
Therefore, an \abr{ETCS} subset is identified and its realization in the redundant system is presented.
Finally, a way of exerting \abr{DDS} to implement a hot standby is described.

%\paragraph{\autoref{cpt:evaluation}}
%TODO

