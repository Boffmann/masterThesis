\chapter{Introduction}
	
Railway operation is undergoing a rapid change.
Ongoing developments in computation and railway applications have improved interlocking technologies, train length and travel speed.
Further, with the introduction of new communication technologies, such as 5G, new possibilties of transmitting information have emerged.
With improvements in railway technologies, the significance of \gls*{ATO} systems to monitor and control a train's operating procedures rises~\cite{YIN2017RNDofATO}.
This has not only the potential to decrease running expenses, but could also help to disburden drivers.
\\

Nowadays, multiple \gls*{ATO} systems have been developed in different countries.
In order to solve the thereby caused incompatabilities among the different \gls*{ATO} systems, \gls*{ETCS}, as a part of \gls*{ERTMS}, has been established as an international standard for \gls*{ATO} systems~\cite{ETCS26}.
Because of the railway domain being a safety-critical environment, a system solving the \gls*{ETCS} specification must comply with certification requirements, such as safety and reliability characteristics, to be applicable for practical use.

The compliance of system requirements is ensured during an approval process.
The thereby applied verification, validation and certification steps are very time- and cost-intensive.
Moreover, the approval process needs to be repleated every time the system updates.
\\

A solution for simplifying the approval process is to combine related functions into modules and map these modules onto a distributed system.
A benefit of this concept is, that it allows to build safe and reliable systems out of less safe and less reliable parts.
%(ref slides fault tolerance).
The distribution approach is already applied in modern safety-critical systems, but also introduces new challenges such as handling the communication overhead and dealing with node-, network and computation failures~\cite{DistributedSafety2020}.
Redundancy, recovery and protection mechanisms are used to achieve reliability and safety~\cite{ChakrabortyFaultTolerantRailway}. 
In order to facilitate a safe and reliable communication among participants, the \gls*{DDS} standard, which follows a publish-subscribe communication pattern, can be used.
\Gls*{DDS} middlewares features real-time support, safety, security and good interoperability with other middlewares by providing various \glspl*{QOS}~\cite{DistributedSafety2020}.
The middlewares are already sucessfully applied in safety-critical environments such as underground railways~\cite{DDSInSubways} and automotive~\cite{DistributedSafety2020}.
Typically, only a subset of \gls*{DDS} is required for a certain use case.
However, state of the art \gls*{DDS} implementations, such as Vortex OpenSplice by ADLINK, cover the entire \gls*{DDS} specification.
Thus, in order to reduce the approval costs of a specific safety-critical application, a subset of all \gls*{DDS}-features needs to be identified.


\section{Contribution}


In this work, approaches for building a safe and reliable distributed system that applies \gls*{DDS} on its middleware layer for communication, are presented.
A key technique for increasing a system's reliability and safety in distributed systems is through redundancy~\cite{BarryFaultToleranceAnalysis}.
Therefore, in this work, proven redundancy patterns are analyzed towards their compatibility with the publish-subscribe pattern of \gls*{DDS}.
Further, a subset of the \gls*{DDS} features and \gls*{QOS}, suitable for solving each presented architecture, is identified.
Finally, the findings are presented in a practical implementation following a selected part of \gls*{ETCS}.

\section{Outline}

\paragraph{\autoref{chptr:concepts}}
In this chapter, an overview about popular patterns and characteristics for building safe and reliable systems, as well as about \gls*{DDS}, is given.
Afterwards, mathematical concepts for evaluating the safety and reliability of these systems, are pointed out.
Finally, however, a selection of related work that uses \gls{DDS} in safety critical applications, is presented.

\paragraph{Systems Evaluation}
In this chapter, an example scenario and its requirements are presented.
Therefore, a risk analysis is made, a data model is presented and a high level analysis towards reliability and safety is conducted.
Finally, different possible system patterns are presented and compared based on their reliability, practicability and comparability with the \gls*{DCPS}.

\paragraph{Implementation}
In this chapter, the practical implementation of an example architecture using hardware redundancy and \gls*{DDS}'s publish-subscribe pattern, as well as neccessary \glspl*{QOS}, are presented.

\paragraph{Evaluation}
Here, the practical system is evaluated and a conclusion is made.
